\chapter{Introducci'on}

La aparición de ambientes de computación centrados en la nube, los cuales se caracterizan por ofrecer servicios bajo demanda; ha favorecido el desarrollo de diversas herramientas que apoyan los procesos de aprovisionamiento en demanda de servicios y ambientes de computación orientados al procesamiento de tareas de larga duración y manejo de grandes volúmenes de datos. Estos ambientes dinámicos de computación son desarrollados mayormente a través de técnicas de programación ágil las cuales se caracterizan por ofrecer rápidos resultados e integración a gran escala de componentes de software. Es así como los equipos de DevOps \footnote{DevOps consiste en traer las prácticas del desarrollo ágil a la administración de sistema y el trabajo en conjunto entre desarrolladores y administradores de sistemas. DevOps no es una descripción de cargo o el uso de herramientas, sino un método de trabajo enfocado a resultados.} se convierten en un elemento fundamental ya que potencia la estabilidad y uniformidad de los distintos ambientes de prueba y producción de modo que los procesos de integración y despliegue se hagan de forma mas automatizada. \\
\\
Las herramientas de aprovisionamiento automático de infraestructura son eje central de estos equipos ya que es a través de ellas que el personal de desarrollo y operaciones son capaces de hablar un mismo lenguaje y establecer los requerimientos y necesidades a satisfacer. Sin embargo, las herramientas actuales de aprovisionamiento adolecen de servicios que faciliten la especificación de infraestructura a través de un API \footnote{ API: Application Programming Interface, conjunto de subrutinas, funciones y procedimientos que ofrece un software para ser utilizado por otro software como una capa de abstracción.} estandarizado que posibilite la orquestación del despliegue de infraestructura a través de Internet.\\
\\
En este documento se presenta una herramienta de aprovisionamiento con orientación a servicios que permite el despliegue y orquestación de plataformas y servicios a través de un API RESTful \footnote{RESTful hace referencia a un servicio web que implementa la arquitectura REST}.\\
\\
Además del capítulo de introducción, la tesis consta de otros 8 capítulos y de 3 apéndices. A continuación se describe brevemente el contenido de cada uno de ellos.\\
\\
En el capítulo 2 se presenta el estado del arte correspondiente a los tipos de virtualización existentes y las diferentes herramientas que existen en el mercado para realizar aprovisionamiento virtual.

...

En el apendice A, ...
