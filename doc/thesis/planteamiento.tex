\chapter{Planteamiento del problema}
\label{capproblema}

\section{Problema}
El aprovisionamiento de recursos empresariales (p.e. servidores web y servidores de bases de datos) es un problema moderamente entendido y abordado pero poco se ha estudiado el aprovisionamiento integral no solo de recursos computacionales sino tambi'en de servicios, plataformas e infraestructuras.\\
\\
Actualmente, con la creciente adopci'on de modelos de computaci'on como el \emph{Cloud Computing} y el \emph{Grid Computing} por parte de las empresas, los ambientes computacionales se han tornado cada vez m'as sofisticados y complejos, requiriendo una solucion computacional que trate de manera integral el f'acil aprovisionamiento de diferentes servicios (SaaS, PaaS, IaaS) sobre ambientes virtuales capaces de atender la variable demanda computacional a trav'es del despliegue de eco-infraestructuras el'asticas de computaci'on.\\
\\
Hoy en d'ia, se encuentran diversas soluciones que abordan mayormente el problema de aprovisionamiento de \emph{enterprise appliances} pero estas dif\'icilmente incorporan  de manera unificada otros servicios tales como infraestructuras, plataformas y software.\\
\\
En consecuencia, para lograr un aprovisionamiento integral en los actuales modelos de computaci'on, este proyecto propone como objetivo principal, plantear el dise'no de un framework computacional cuyas partes soporten una arquitectura que permita un eficiente aprovisionamiento de software de manera autom'atica, para ambientes virtualizados. De igual modo, se propone realizar una ejemplificaci'on del framework en un ambiente virtualizado.\\
\\
Para lograrlo ser'a necesario evaluar diferentes estilos arquitect'onicos, realizar un meta-modelo del framework, definir la plataforma en la que se realizara una ejemplificaci'on y definir el mecanismo de aprovisionamiento que se utilizara para aprovisionar ambientes en maquinas virtuales.\\ 
\\
Se ha estimado que el proyecto se realice en un periodo de seis meses. En los siguientes cap'itulos se detallaran los objetivos del proyecto, las tecnolog'ias actuales que est'an trabajando alrededor del tema, la metodolog'ia y actividades detalladas que se realizaran para alcanzar los objetivos planteados.

\section{Objetivo General}
Dise'nar un framework de aprovisionamiento de software autom'atico, para ambientes virtualizados.

\section{Objetivos Espec'ificos}
\begin{itemize}
\item Evaluar diferentes estilos arquitecturales que soporten el framewok de aprovisionamiento.
\item Evaluar diferentes mecanismos de aprovisionamiento que se utilizan en la actualidad.
\item Seleccionar la plataforma a la que va orientada la arquitectura del framewok.
\item Evaluar los requerimientos necesarios para montar clusters en maquinas virtuales de manera simplificada.
\item Dise'nar planes de prueba para ambientes virtuales y f'isicos que permitan comparar el m'etodo propuesto con los m'etodos manuales.
\end{itemize}
