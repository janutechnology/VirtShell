\chapter{Arquitectura y Diseño}
\label{Arquitectura}

El VirtShell Framework es un API REST diseñado para simplificar la automatización y gestión de infraestructura, facilitando tareas como creación, despliegue y mantenimiento de recursos virtuales. Para lograr esto, VirtShell se basa en Shell scripts re-utilizables que permiten instalar y configurar cualquier tipo de servidor o grandes soluciones que involucren varios recursos virtuales sin importar su tamaño. VirtShell es un framework de código abierto y bajo la licencia BSD, que permite utilizarlo para proyectos de cualquier tipo, incluso comerciales. \\
\\
\section{Características}

\begin{description}
\item [Programable] VirtShell esta principalmente orientado a realizar el aprovisionamiento de sus instancias en scripts de shell permitiendo aprovechar todas las estructuras y utilidades del lenguaje de programacion de shell. Adicionalmente VirtShell permite extender el comportamiento del shell desplegando comandos propios que proporcionan ahorro en tiempo y en complejidad. Sin embargo, el lenguaje de shell no es de uso obligatorio, el  metodo de aprovisionamiento puede ser el que el usuario desee. 
\item [Reutilizable] VirtShell ofrece las herramientas para que los scritps de aprovisionamiento sean configurables y que puedan ser ejecutados varias veces en diferentes ambientes de desarrollo o produccion.
\item [Modular] VirtShell es un framework organizado de forma modular, a pesar de que cuenta con un gran número de paquetes y clases. Los modulos seran explicados ampliamente en este capitulo mas adelante.
\item [Seguro] VirtShell provee varias capacidades y servicios para aumentar la privacidad y el control de acceso a los diferentes recursos. Los servicios de seguridad permiten crear redes y controlar el acceso a las instancias creadas, asi como definir y administrar politicas de accesso a usuarios y permisos sobre cualquier recurso del sistema como por ejemplo scripts de creacion y aprovisionamiento.
\item [Extensible] VirtShell fue diseñado con la idea de cargar codigo dinamicamente facilmente, permitiendo extender el comportamiento del framework agregando plugins en tiempo de ejecucion. 
\item [Inyección de dependencias virtuales] VirtShell adopta la idea del patrón de Inyección de Dependencias, para conseguir scripts de aprovisionamiento mas desacoplados, facilitando a un recurso virtual configurar las dependencias que tiene de otras máquinas virtuales para realizar su trabajo. Para ello, el framework permiten declarar el listado de dependencias de recursos virtuales que tiene un script aprovisionamiento permitiendo el correcto acople entre los diferentes recursos virtuales. 
\end{description}

\section{Arquitectura}
...

\subsection{Agentes}
....

