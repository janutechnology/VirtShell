\chapter{Arquitectura y Diseño}
\label{Arquitectura}

VirtShell Framework es una plataforma que proporciona herramientas para la automatización y gestión de infraestructura a través del protocolo HTTP. En otros términos, VirtShell facilita la creación, despliegue, mantenimiento y monitoreo tanto de recursos virtuales como físicos por medio de una API REST. Esto permite que cualquier desarrollo de software con acceso a internet (sitio web, aplicación móvil, etc.) pueda utilizar VirtShell e interactuar con la infraestructura tan solo haciendo llamadas a direcciones de internet. \\
\\
VirtShell basa principalmente su funcionamiento en scripts de shell re-utilizables, que facilitan la instalación y configuración de cualquier tipo de servidor o grandes soluciones que involucren varios recursos virtuales sin importar su tamaño. Sin embargo el lenguaje de los scripts no necesariamente debe ser el shell también puede interactuar con cualquier lenguaje de programación que el usuario prefiera.\\
\\
VirtShell es un framework de código abierto y bajo la licencia BSD, que permite utilizarlo para proyectos de cualquier tipo, incluso comerciales. 

\section{Características}

\begin{description}
\item [Programable] VirtShell esta orientado a realizar el aprovisionamiento de sus instancias principalmente por medio de scripts escritos en shell, permitiendo aprovechar todas las estructuras y utilidades del lenguaje de programación. Sin embargo, el lenguaje de shell no es de uso obligatorio, el  metodo de aprovisionamiento puede ser el de la preferencia del usuario. 
\item [Repetible] VirtShell ofrece herramientas para que los scritps de aprovisionamiento sean configurables y  puedan ser ejecutados varias veces en diferentes ambientes de desarrollo o producción.
\item [Modular] VirtShell es un framework organizado de forma modular. Los módulos se encuentran agrupados en categorías que ofrecen las herramientas necesarias para la administración de y aprovisionamiento de múltiples recursos virtuales. De igual manera y dada las características de REST, los módulos están diseñados para que se puedan dividir en diferentes servidores obteniendo "micro APIs", lo que permite dividir los procesos y atender diferentes tipos de operaciones del API. 
\item [Escalable] Al ser VirtShell de código abierto, el API puede modificarse, crecer fácilmente y versionarse de diferentes maneras. 
\item [Seguro] VirtShell provee varias capacidades y servicios para aumentar la privacidad y el control de acceso a los diferentes recursos. Los servicios de seguridad permiten crear redes y controlar el acceso a las instancias creadas, así como definir y administrar políticas de acceso a usuarios y permisos sobre cualquier recurso del sistema como por ejemplo scripts de creación y aprovisionamiento.
\item [Extensible] VirtShell fue diseñado con la idea de cargar código dinámicamente fácilmente, permitiendo extender el comportamiento del framework agregando plugins en tiempo de ejecución.  Adicionalmente VirtShell permite extender el comportamiento del shell desplegando comandos propios que proporcionan ahorro en tiempo y en complejidad.
\item [Inyección de dependencias virtuales] VirtShell adopta la idea del patrón de Inyección de Dependencias, para conseguir scripts de aprovisionamiento mas desacoplados. De esta manera facilita las configuración de las dependencias que tiene un recurso virtual de otras máquinas virtuales. Para ello, el framework permiten declarar el listado de dependencias de recursos virtuales que tiene un script aprovisionamiento encargándose del correcto acople entre los diferentes recursos virtuales.
\item [Interoperable] Al seguir el estilo arquitectónico REST y contar con la documentación detallada sobre cada uno de los recursos y urls que expone el API de VirtShell, se logra una capacidad clave para la administración remota de la infraestructura virtual. Esta capacidad se refiere al hecho de poder desarrollar aplicaciones en cualquier plataforma y para cualquier dispositivo electrónico, lo que permite funcionar con otros productos o sistemas existentes o futuros.

\end{description}

\section{Arquitectura}
VirtShell Framework consiste de características organizadas en 13 módulos. Estos módulos son agrupados en Seguridad, Administración y Aprovisionamiento. Estos elementos se usan de manera separada pero trabajan juntos para proveer la información necesaria para que los agentes realicen su trabajo en los hosts que albergaran los recursos virtuales, como se muestra en la figura. \\

\begin{figure}
	\caption{Overview of the VirtShell Framework}
	\includegraphics[width = 0.8\textwidth]{../architecture/v1/diagrams/framework}
\end{figure}

Las siguientes secciones detallan los módulos disponibles para cada característica. 

\subsection{Security}
Seguridad consiste de los módulos de usuarios (users), grupos (groups) y el modulo de autenticación (authenticator). El control de los usuarios y grupos son elementos clave en la administración del framework. Los \textbf{Usuarios} pueden ser personas reales, es decir, cuentas ligadas a un usuario físico en particular o cuentas que existen para ser usadas por aplicaciones específicas. \\
\\
Los \textbf{Grupos} son expresiones lógicas de organización, reuniendo usuarios para un propósito común. Los usuarios dentro de un mismo grupo pueden leer, escribir o ejecutar los recursos que pertenecen a ese grupo.\\
\\
El módulo de \textbf{autenticación} soporta el proceso por el cual cuando un usuario se presenta a la aplicación puede validar su identidad es, de hecho, quien decide si tiene permiso para ingresar al sistema y el nivel de acceso a un recurso dado. En el capitulo 4 se detalla el proceso de autenticación y autorización.

\subsection{Managment}
Administración consiste de los modulos anfitriónes (hosts), particiones (partitions), ambientes (enviroments), instancias (instances), propiedades (properties) y tareas (tasks). \\
\\
El módulo de \textbf{anfitriónes} lleva registro de nodos físicos, servidores o máquinas virtuales, que se encuentren conectados a la red y que permitan albergar instancias virtuales. Los anfitriónes son clasificados de acuerdo a diferentes combinaciones de CPU, memoria, almacenamiento y capacidad de trabajo en red, dando flexibilidad para elegir la opcion mas adecuada para las necesidades de las aplicaciones destino. En otras palabras el tipo de anfitrión determina el hardware del nodo físico que sera usado por los recursos virtuales.\\
\\
El módulo de \textbf{particiones} permite dividir los anfitriones en secciones aisladas de disponibilidad. Cada partición puede ser usada para definir áreas geográficas separadas o simplemente para dividir los nodos físicos en subgrupos destinados a diferentes usos o equipos de trabajo o tipos de clientes.\\
\\
El módulo de \textbf{ambientes} permite dividir lógicamente una partición en subredes de trabajo más pequeñas, con lo que se crean grupos más pequeños con diferentes fines. En Los ambientes de trabajo se configuran los usuarios que tienen permiso para interactuar trabajar en el.\\
\\
El módulo de \textbf{instancias} es un recurso virtual o máquina virtual o contenedor con parámetros y capacidades definidas en la cual se puede instalar el software deseado. Un usuario puede crear, aprovisionar, actualizar y finalizar instancias en VirtShell tanto como necesite dando la sensación de elasticidad de la red.\\
\\
El módulo de \textbf{propiedades} permite consultar información de sistema, de las instancias o de los anfitriones físicos. La información que puede ser consultada es toda aquella que el sistema tenga disponible o que se pueda consultar por medio de comandos de sistema o comandos propios de VirtShell. Ejemplos de información que se puede consultar por medio de las propiedades es porcentajes de memoria y cpu usada, numero de procesos en ejecución, etc. Las propiedades pueden ser consultadas en una sola maquina o simultáneamente a varias maquinas o a un conjunto de maquinas de acuerdo a prefijos en su nombre.\\
\\
Finalmente, el módulo de \textbf{tareas} da la información y estado sobre las diferentes tareas o trabajos ejecutados en el sistema. Debido a que el medio para interactuar con los módulos de VirtShell es a través de un API REST, una petición de creación de un nuevo recurso virtual, puede tomar algo de tiempo, y puede ser un tiempo largo, si el aprovisionamiento involucra varias maquinas. Para evitar, tener una petición esperando respuesta, VirtShell crea una tarea que sera ejecutada de manera asincrónica, dando como respuesta, un identificador de la tarea, para que esta pueda ser consultada posteriormente y conocer el estado de la petición.


\subsection{Provisioning}
Aprovisionamiento consiste de los módulos aprovisonadores (provisioners), imágenes (images), archivos (files) y paquetes (packages).

\subsection{Agents}

