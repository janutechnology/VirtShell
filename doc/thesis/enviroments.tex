\subsection{Hosts}
Representan subredes de trabajo más pequeñas asociadas a una partición. Los métodos soportados son:

\begin{center}
 \captionof{table}{Métodos HTTP para enviroments}
 \begin{tabular}{| l | l | l | l |}
 \hline
  \rowcolor{blueapi}
  \textbf{Acción} & \textbf{Método HTTP} & \textbf{Solicitud HTTP} & \textbf{Descripción} \\ [0.5ex] 
  \hline\hline
  get & GET & /enviroments/:name & Gets one enviroment by name. \\
  \hline
  list & GET & /enviroments & Retrieves the list of enviroments. \\
  \hline  
  create & POST & /enviroments/ & Inserts a new enviroment configuration. \\
  \hline
  delete & DELETE & /enviroments/:name & Deletes an existing enviroment. \\
  \hline
\end{tabular}
\end{center}

Representaci'on del recurso de un ambiente:

\medskip
\begin{lstlisting}[style=json]
{
  "uuid": string,
  "name": string,
  "description": string, 
  "users": [ user_resource],
  "partition": string,
  "created": {"at": number, "by": string},
  "modified": {"at": number, "by": string}
}
\end{lstlisting}

Ejemplo:

\medskip
\begin{lstlisting}[style=json]
{
  "uuid": "ab8076c0-db91-11e2-82ce-0002a5d5c51b",
  "name": "bigdata_test_01",
  "description": "Collection of servers oriented to big data.", 
  "users": [ ... list of users allowed to use the enviroment ...],
  "partition": "partition associated with the enviroment",
  "created": {"at":"1429207233", "by":"92d30f0c-8c9c-11e5-8994-feff819cdc9f"},
  "modified": {"at":"1529207233", "by":"92d31132-8c9c-11e5-8994-feff819cdc9f"}
}
\end{lstlisting}

\subsubsection{Ejemplos de peticiones HTTP}

\paragraph{Crear un nuevo ambiente - POST /api/virtshell/v1/enviroments} ~\\

\begin{lstlisting}[style=json]
curl -sv -X POST \
  -H 'accept: application/json' \
  -H 'X-VirtShell-Authorization: UserId:Signature' \
  -d '{
       "name": "bigdata_test_01",
       "description": "Collection of servers oriented to big data.", 
       "users": [ ... list of users allowed to use the enviroment ...],
       "partition": "partition associated with the enviroment"
      }' \
   'http://localhost:8080/api/virtshell/v1/enviroments'
\end{lstlisting}

Response:

\begin{lstlisting}[style=json]
HTTP/1.1 200 OK
Content-Type: application/json
{ "create": "success" }
\end{lstlisting}

\paragraph{Obtener un ambiente- GET /api/virtshell/v1/enviroments/:name} ~\\

\begin{lstlisting}[style=json]
curl -sv -H 'accept: application/json' 
     -H 'X-VirtShell-Authorization: UserId:Signature' \ 
     'http://<host>:<port>/api/virtshell/v1/enviroments/backend_development'
\end{lstlisting}

Response:

\begin{lstlisting}[style=json]
HTTP/1.1 200 OK
Content-Type: application/json
{
  "uuid": "efa1777c-cad7-11e5-9956-625662870761",
  "name": "backend_development",
  "description": "All backend of the company", 
  "users": [ ... list of users allowed to use the enviroment ...],
  "partition": "partition associated with the enviroment",
  "created": {"at":"1429207233", "by":"1a900cdc-cad8-11e5-9956-625662870761"},
  "modified": {"at":"1529207233", "by":"2163b554-cad8-11e5-9956-625662870761"}
}
\end{lstlisting}

\paragraph{Obtener todos los ambientes - GET /api/virtshell/v1/enviroments} ~\\

\begin{lstlisting}[style=json]
curl -sv -H 'accept: application/json' 
     -H 'X-VirtShell-Authorization: UserId:Signature' \ 
     'http://localhost:8080/api/virtshell/v1/enviroments'
\end{lstlisting}

Response:

\begin{lstlisting}[style=json]
HTTP/1.1 200 OK
Content-Type: application/json
{
  "enviroments": [
    {
      "uuid": "ab8076c0-db91-11e2-82ce-0002a5d5c51b",
      "name": "bigdata_test_01",
      "description": "Collection of servers oriented to big data.", 
      "users": [ ... list of users allowed to use the enviroment ...],
      "partition": "partition associated with the enviroment",
      "created": {"at":"1429207233", "by":"92d30f0c-8c9c-11e5-8994-feff819cdc9f"},
      "modified": {"at":"1529207233", "by":"92d31132-8c9c-11e5-8994-feff819cdc9f"}
    },
    { 
      "uuid": "efa1777c-cad7-11e5-9956-625662870761",
      "name": "backend_development",
      "description": "All backend of the company", 
      "users": [ ... list of users allowed to use the enviroment ...],
      "partition": "partition associated with the enviroment",      
      "created": {"at":"1429207233", "by":"1a900cdc-cad8-11e5-9956-625662870761"},
      "modified": {"at":"1529207233", "by":"2163b554-cad8-11e5-9956-625662870761"}
    }    
  ]
}   
\end{lstlisting}

\paragraph{Eliminar un ambiente - DELETE /api/virtshell/v1/enviroments/:name} ~\\

\begin{lstlisting}[style=json]
curl -sv -X DELETE \
   -H 'accept: application/json' \
   -H 'X-VirtShell-Authorization: UserId:Signature' \
   'http://<host>:<port>/api/virtshell/v1/enviroments/backend_development'
\end{lstlisting}

Response:

\begin{lstlisting}[style=json]
HTTP/1.1 200 OK
Content-Type: application/json
```
```json
{ "delete": "success" }
\end{lstlisting}