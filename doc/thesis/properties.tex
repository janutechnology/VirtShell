\subsection{Properties}
Representan propiedades de configuraci'on de las m'aquinas virtuales o contenedores. Los metodos soportados son:

\begin{center}
 \begin{tabular}{| l | l | l | l |}
 \hline
  \rowcolor{blueapi}
  \textbf{Acci'on} & \textbf{Metodo HTTP} & \textbf{Solicitud HTTP} & \textbf{Descripci'on} \\ [0.5ex] 
  \hline\hline
  get & GET & /properties/ & Install one or more packages. \\ [1ex] 
  \hline
\end{tabular}
\end{center}

\vspace{1cm}
Representaci'on del recurso de un paquete:
\vspace{1cm}

\begin{lstlisting}[style=json]
{
  "properties": [
      {"name": "propertie_name1"},
      {"name": "propertie_name2"}
  ],
  "hosts": [ 
      {"name": "Host_", "range": "[1-3]"}, 
      {"name": "database_001"}
  ],
  "tags": [
    {"name": "db"},
    {"name": "web"}
  ]
}
\end{lstlisting}

Ejemplo:

\medskip
\begin{lstlisting}[style=json]
{
  "properties": [
      {"name": "memory"},
      {"name": "cpu"}
  ],
  "hosts": [ 
      {"name": "Host_", "range": "[1-3]"}
  ]
}
\end{lstlisting}

\subsection{Ejemplos de peticiones HTTP}

\subsubsection{Obtener una o mas propiedades de una unica instancia - POST /virtshell/api/v1/properties}

\begin{lstlisting}[style=json]
curl -sv -X GET \
  -H 'accept: application/json' \
  -H "Content-Type: text/plain" \
  -H 'X-VirtShell-Authorization: UserId:Signature' \
  -d '{ "properties": [{"name": "memory"}, {"name": "cpu"}],
        "hosts": [{"name": "WebServer"}]}' \
   'http://localhost:8080/api/virtshell/v1/properties'
\end{lstlisting}

\vspace{1cm}
Respuesta:
\vspace{1cm}

\begin{lstlisting}[style=json]
HTTP/1.1 202 OK
Content-Type: application/json
{
  "id": "kj5436c0-dc94-13tg-82ce-9992b5d5c51b",
  "name": "Database001",
  "memory": 1024
}
\end{lstlisting}

\subsubsection{Obtener una o mas propiedades de una o mas instancias por tag - POST /virtshell/api/v1/properties}

\begin{lstlisting}[style=json]
curl -sv -X GET \
  -H 'accept: application/json' \
  -H "Content-Type: text/plain" \
  -H 'X-VirtShell-Authorization: UserId:Signature' \
  -d '{ "properties": [{"name": "memory"}, {"name": "cpu"}],
        "tag": [{"name": "web"}]}' \
   'http://localhost:8080/api/virtshell/v1/properties'
\end{lstlisting}

\vspace{1cm}
Respuesta:
\vspace{1cm}

\begin{lstlisting}[style=json]
HTTP/1.1 202 OK
Content-Type: application/json
{
  properties: [
    {
     "id": "kj5436c0-dc94-13tg-82ce-9992b5d5c51b",
     "name": "WebServerPhp001",
     "memory": 1024,
     "cpu": 2
    },
    {
     "id": "591b3828-7aaf-4833-a94c-ad0df44d59a4",
     "name": "WebServerPhp002",
     "memory": 1024,
     "cpu": 1  
    }
  ]
}
\end{lstlisting}

\subsubsection{Obtener una o mas propiedades de una o mas instancias usando como prefijo un rango - POST /virtshell/api/v1/properties}

\begin{lstlisting}[style=json]
curl -sv -X GET \
  -H 'accept: application/json' \
  -H "Content-Type: text/plain" \
  -H 'X-VirtShell-Authorization: UserId:Signature' \
  -d '{ "properties": [{"name": "memory"}, {"name": "cpu"}],
        {"name": "Database00", "range": "[1-3]"}]}' \
   'http://localhost:8080/api/virtshell/v1/properties'
\end{lstlisting}

\vspace{1cm}
Respuesta:
\vspace{1cm}

\begin{lstlisting}[style=json]
HTTP/1.1 202 OK
Content-Type: application/json
{
  properties: [
    {
     "id": "kj5436c0-dc94-13tg-82ce-9992b5d5c51b",
     "name": "Database001",
     "memory": 4024,
     "cpu": 2
    },
    {
     "id": "591b3828-7aaf-4833-a94c-ad0df44d59a4",
     "name": "Database002",
     "memory": 4024,
     "cpu": 1  
    },
    {
     "id": "f7c81039-5c88-423b-8b0d-c124483d586b",
     "name": "Database003",
     "memory": 4024,
     "cpu": 3  
    }
  ]  
}
\end{lstlisting}