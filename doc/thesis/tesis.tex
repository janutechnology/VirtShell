\documentclass[oneside,numbers,spanish]{ezthesis}
%% # Opciones disponibles para el documento #
%%
%% Las opciones con un (*) son las opciones predeterminadas.
%%
%% Modo de compilar:
%%   draft            - borrador con marcas de fecha y sin im'agenes
%%   draftmarks       - borrador con marcas de fecha y con im'agenes
%%   final (*)        - version final de la tesis
%%
%% Tama'no de papel:
%%   letterpaper (*)  - tama'no carta (Am'erica)
%%   a4paper          - tama'no A4    (Europa)
%%
%% Formato de impresi'on:
%%   oneside          - hojas impresas por un solo lado
%%   twoside (*)      - hijas impresas por ambos lados
%%
%% Tama'no de letra:
%%   10pt, 11pt, o 12pt (*)
%%
%% Espaciado entre renglones:
%%   singlespace      - espacio sencillo
%%   onehalfspace (*) - espacio de 1.5
%%   doublespace      - a doble espacio
%%
%% Formato de las referencias bibliogr'aficas:
%%   numbers          - numeradas, p.e. [1]
%%   authoryear (*)   - por autor y a'no, p.e. (Newton, 1997)
%%
%% Opciones adicionales:
%%   spanish         - tesis escrita en espa'nol
%%
%% Desactivar opciones especiales:
%%   nobibtoc   - no incluir la bibiolgraf'ia en el 'Indice general
%%   nofancyhdr - no incluir "fancyhdr" para producir los encabezados
%%   nocolors   - no incluir "xcolor" para producir ligas con colores
%%   nographicx - no incluir "graphicx" para insertar gr'aficos
%%   nonatbib   - no incluir "natbib" para administrar la bibliograf'ia

%% Paquetes adicionales requeridos se pueden agregar tambi'en aqu'i.
%% Por ejemplo:
%\usepackage{subfig}
%\usepackage{multirow}

%% # Datos del documento #
%% Nota que los acentos se deben escribir: \'a, \'e, \'i, etc.
%% La letra n con tilde es: \~n.


\usepackage{titlesec}
\usepackage{fancyhdr}
\usepackage[Sonny]{fncychap} %Options: Sonny, Lenny, Glenn, Conny, Rejne
                             %         Bjarne, Bjornstrup
\usepackage{lmodern}
\usepackage{listings}
\usepackage{color}
\usepackage{array}
\usepackage{colortbl}
\usepackage{url}
\usepackage{xcolor}
\usepackage[utf8]{inputenc}


\setcounter{secnumdepth}{4}

\author{Carlos Alberto Llano Rodríguez}
\title{VirtShell - Framework para aprovisionamiento de soluciones virtuales}
\degree{Maestría en Ingeniería con énfasis en Ingeniería de Sistemas}
\supervisor{Jhon Alexander Sanabria}
\institution{Universidad del Valle}
\faculty{Escuela de Ingeniería de Sistemas y Computación}
\department{Facultad de Ingeniería}

%% # M'argenes del documento #
%% 
%% Quitar el comentario en la siguiente linea para austar los m'argenes del
%% documento. Leer la documentaci'on de "geometry" para m'as informaci'on.

%\geometry{top=40mm,bottom=33mm,inner=40mm,outer=25mm}

%% El siguiente comando agrega ligas activas en el documento para las
%% referencias cruzadas y citas bibliogr'aficas. Tiene que ser *la 'ultima*
%% instrucci'on antes de \begin{document}.
\hyperlinking


% Colors in http://latexcolor.com/
\definecolor{lightgray}{rgb}{.9,.9,.9}
\definecolor{darkgray}{rgb}{.4,.4,.4}
\definecolor{purple}{rgb}{0.65, 0.12, 0.82}
\definecolor{cornflowerblue}{rgb}{0,0.4,0.8}
\definecolor{blueapi}{rgb}{0.74, 0.83, 0.9}
\definecolor{magnolia}{rgb}{0.97, 0.96, 1.0}
\colorlet{punct}{red!60!black}
\definecolor{background}{HTML}{EEEEEE}
\definecolor{delim}{RGB}{20,105,176}
\colorlet{numb}{magenta!60!black}
\definecolor{navyblue}{rgb}{0.0, 0.0, 0.5}

\newcommand\JSONnumbervaluestyle{\color{navyblue}}
\newcommand\JSONstringvaluestyle{\color{red}}

% switch used as state variable
\newif\ifcolonfoundonthisline

\makeatletter

\lstdefinestyle{json}
{
  showstringspaces    = false,
  numbers             = left,
  numberstyle         = \scriptsize,
  stepnumber          = 1,
  numbersep           = 8pt,
  breaklines          = true,
  frame               = lines,
  alsoletter          = 0123456789.,
  morestring          = [s]{"}{"},
  literate            = {:}{{{\color{punct}{:}}}}{1}
                        {,}{{{\color{punct}{,}}}}{1}
                        {\{}{{{\color{delim}{\{}}}}{1}
                        {\}}{{{\color{delim}{\}}}}}{1}
                        {[}{{{\color{delim}{[}}}}{1}
                        {]}{{{\color{delim}{]}}}}{1}
                        {|}{{{\color{delim}{|}}}}{1},
  stringstyle         = \ifcolonfoundonthisline\JSONstringvaluestyle\fi,
  MoreSelectCharTable =%
    \lst@DefSaveDef{`:}\colon@json{\processColon@json},
  basicstyle          = \ttfamily,
  backgroundcolor     = \color{magnolia},
  keywordstyle        = \ttfamily\bfseries,
}

% flip the switch if a colon is found in Pmode
\newcommand\processColon@json{%
  \colon@json%
  \ifnum\lst@mode=\lst@Pmode%
    \global\colonfoundonthislinetrue%
  \fi
}

\lst@AddToHook{Output}{%
  \ifcolonfoundonthisline%
    \ifnum\lst@mode=\lst@Pmode%
      \def\lst@thestyle{\JSONnumbervaluestyle}%
    \fi
  \fi
  %override by keyword style if a keyword is detected!
  \lsthk@DetectKeywords% 
}

% reset the switch at the end of line
\lst@AddToHook{EOL}%
  {\global\colonfoundonthislinefalse}

\makeatother

\begin{document}

%% En esta secci'on se describe la estructura del documento de la tesis.
%% Consulta los reglamentos de tu universidad para determinar el orden
%% y la cantidad de secciones que debes de incluir.

%% # Portada de la tesis #
%% Mirar el archivo "titlepage.tex" para los detalles.
%% ## Construye tu propia portada ##
%% 
%% Una portada se conforma por una secuencia de "Blocks" que incluyen
%% piezas individuales de informaci'on. Un "Block" puede incluir, por
%% ejemplo, el t'itulo del documento, una im'agen (logotipo de la universidad),
%% el nombre del autor, nombre del supervisor, u cualquier otra pieza de
%% informaci'on.
%%
%% Cada "Block" aparece centrado horizontalmente en la p'agina y,
%% verticalmente, todos los "Blocks" se distruyen de manera uniforme 
%% a lo largo de p'agina.
%%
%% Nota tambi'en que, dentro de un mismo "Block" se pueden cortar
%% lineas usando el comando \\
%%
%% El tama'no del texto dentro de un "Block" se puede modificar usando uno de
%% los comandos:
%%   \small      \LARGE
%%   \large      \huge
%%   \Large      \Huge
%%
%% Y el tipo de letra se puede modificar usando:
%%   \bfseries - negritas
%%   \itshape  - it'alicas
%%   \scshape  - small caps
%%   \slshape  - slanted
%%   \sffamily - sans serif
%%
%% Para producir plantillas generales, la informaci'on que ha sido inclu'ida
%% en el archivo principal "tesis.tex" se puede accesar aqu'i usando:
%%   \insertauthor
%%   \inserttitle
%%   \insertsupervisor
%%   \insertinstitution
%%   \insertdegree
%%   \insertfaculty
%%   \insertdepartment
%%   \insertsubmitdate

\begin{titlepage}
  \TitleBlock{\scshape\insertinstitution}
  \TitleBlock[\bigskip]{\scshape\insertfaculty}
  \TitleBlock{\Huge\scshape\inserttitle}
  \TitleBlock{\scshape
    Tesis presentada por \insertauthor \\
    para obtener el grado de \insertdegree}
  \TitleBlock{\insertsubmitdate}
  \TitleBlock[\bigskip]{\insertdepartment}
\end{titlepage}

%% Nota 1:
%% Se puede agregar un escudo o logotipo en un "Block" como:
%%   \TitleBlock{\includegraphics[height=4cm]{escudo_uni}}
%% y teniendo un archivo "escudo_uni.pdf", "escudo_uni.png" o "escudo_uni.jpg"
%% en alg'un lugar donde LaTeX lo pueda encontrar.

%% Nota 2:
%% Normalmente, el espacio entre "Blocks" se extiende de modo que el
%% contenido se reparte uniformemente sobre toda la p'agina. Este
%% comportamiento se puede modificar para mantener fijo, por ejemplo, el
%% espacio entre un par de "Blocks". Escribiendo:
%%   \TitleBlock{Bloque 1}
%%   \TitleBlock[\bigskip]{Bloque2}
%% se deja un espacio "grande" y de tama~no fijo entre el bloque 1 y 2.
%% Adem'as de \bigskip est'an tambi'en \smallskip y \medskip. Si necesitas
%% aun m'as control puedes usar tambi'en, por ejemplo, \vspace*{2cm}.




%% # Prefacios #
%% Por cada prefacio (p.e. agradecimientos, resumen, etc.) crear
%% un nuevo archivo e incluirlo aqu'i.
%% Para m'as detalles y un ejemplo mirar el archivo "gracias.tex".
\prefacesection{Agradecimientos}

\begin{itemize}
\item A Dios, por iluminarme en momentos dificiles.
\item A mi familia sin ella, no hubiera podido alcanzar esta meta.
\item A John Alexander Sanabria, profesor de la Facultad de Ingenier'ia de la Universidad del Valle, por su inmensa paciencia y apoyo a lo largo de todo el proyecto.
\item A mis amigos colaboradores de la Universidad del Valle, por todo el apoyo y por creer siempre en este gran esfuerzo.
\item A mis compañeros y amigos por sus palabras de aliento y constante apoyo.
\item A todas aquellas personas que de una u otra forma colaboraron en la realizaci'on del presente trabajo.
\end{itemize}



%% # 'Indices y listas de contenido #
%% Quitar los comentarios en las lineas siguientes para obtener listas de
%% figuras y cuadros/tablas.
\tableofcontents
\listoffigures
\listoftables

%% # Cap'itulos #
%% Por cada cap'itulo hay que crear un nuevo archivo e incluirlo aqu'i.
%% Mirar el archivo "intro.tex" para un ejemplo y recomendaciones para
%% escribir.
\chapter{Introducci'on}

Ideas para la introduccion:

En los últimos años se han producido importantes avances tecnológicos que han dado lugar a la demanda de nuevas aplicaciones relacionadas con la automatización de .........\\
\\
Por un lado ha aumentado el número de aplicaciones de ...... que requieren estrategicas ....... capaces de ...... . Por otro lado, las aplicaciones  de ultima generacion. En esta tesis se propone una estrategia distinta de aprovisionamiento y administracion, que cumplen los requisitos de velocidad y facilidad de uso demandados por los usuarios.\\
\\
Se propone un sistema web rapido y eficaz, capas de ....... para aprovisionar ...... 

basada en el popular metodo de ....., capaz de ......

La estrategia propuesta mejora muy notablemente la ......




\chapter{Aprovisionamiento de recursos virtuales}
\label{aprmaqvir}
La computaci'on en la nube ha sido un punto importante de investigaci'on en la industria recientemente. Esta puede ser descrita como una nueva clase de computaci'on en la cual din'amicos y escalables recursos pueden ser provistos sobre internet. Para los usuarios esto es transparente y ellos solo pagan lo que usan de acuerdo a niveles de servicio establecidos con los proveedores de nubes.\\
\\
Una de las principales caracter'isticas de la computaci'on en la nube es la virtualizaci'on, la cual consiste en crear una versi'on virtual de un recurso tecnologico en lugar de usar una versi'on f'isica. La virtualizaci'on se puede aplicar a computadoras, sistemas operativos, dispositivos de almacenamiento de informaci'on, aplicaciones o redes permitiendo que las empresas ejecuten m'as de un sistema virtual, adem'as de m'ultipls sistemas operativos y aplicaciones, en un 'unico servidor, de esta manera se logra econom'ia de escala y una mayor eficiencia.\\
\\
En la actualidad predominan dos tecnicas de virtualizacion, la primera tecnica se denomina virtualizaci'on de hardware y consiste en crear un hardware sint'etico el cual usan las maquinas virtuales como propio, la idea es virtualizar el sistema operativo completo el cual se ejecuta sobre un software llamado el hipervisor, su funci'on es interactuar directamente con la CPU en el servidor f'isico, ofreciendo a cada uno de los servidores virtuales una total autonom'ia e independencia. Incluso pueden coexistir en una misma m'aquina distintos servidores virtuales funcionando con distintos sistemas operativos. Esta tecnica es la mas desarrollada y hay diferentes clases que cada fabricante ha ido desarrollando y adaptando, como por ejemplo Xen, KVM, VMWare y VirtualBox.\\
\\
La segunda tecnica es conocida como virtualizaci'on del sistema operativo. En esta t'ecnica lo que se virtualiza es el sistema operativo completo el cual corre directamente virtual sobre la m'aquina f'isica. En esta t'ecnica las maquinas virtuales son llamadas contenedores, los cuales acceden por igual a todos los recursos del sistema. La ventaja es a su vez una desventaja: Todas las maquinas virtuales usan el mismo Kernel que el sistema operativo lo que reduce mucho los errores y multiplica el rendimiento, pero a su vez solo puede haber un mismo tipo de sistema operativo en los contenedores, no se puede mezclar Windows-Linux-Etc. Este sistema, tambi'en se acerca mucho a lo que seria una virtualizaci'on nativa.\\
\\
Sin importar la tecnica de virtualizaci'on que se use, la instalaci'on de una maquina virtual (o de un contenedor) requiere normalmente de la generaci'on e instalaci'on de una imagen y la instalaci'on y configuraci'on de paquetes de software. Estas tareas generalmente son realizadas por t'ecnicos de los proveedores de la nube. Cuando un usuario de la nube solicita un nuevo servicio o mas capacidad de computo, el administrador selecciona la apropiada imagen para clonar e instalar en los nodos de la nube. Si no hay una imagen apropiada para los requerimientos del cliente se crea y configura una nueva que cumpla con la solicitud. Esta creaci'on de una nueva imagen puede ser realizada modificando la imagen mas cercada de las ya existentes. En el momento de la creaci'on optima de la imagen un administrador puede tener dificultades y pregunt'as como, cual es la mejor configuraci'on?, cuales paquetes y sus dependencias deber'ian ser instaladas? y como encontrar una imagen que mejor llene las expectativas?.\\
\\
Es por esta raz'on que los proveedores de la nube desean cada vez mas automatizar y simplificar este proceso porque la dependencia entre paquetes de software y la dificultad de mantenimiento agrega tiempo a la creaci'on de las maquinas virtuales. En otras palabras los proveedores de nube quieren dar mas flexibilidad y agilidad a la hora de satisfacer los requerimientos de los usuarios finales.\\
\\
Existen varias soluciones que permiten la interacci'on con los diferentes ambientes de virtualizaci'on. Estas soluciones usan diferentes enfoques para realizar despliegues de software en las maquinas virtuales,  que dan un r'apido, controlado y autom'atico despliegue de software, en todas las maquinas de una red f'isicas o virtualizadas, permitiendo mejorar los tiempos de instalaci'on de nuevas funcionalidades de forma confiable y segura de la misma forma que ayudan a disminuir el tiempo y el costo de los despliegues de aplicaciones y servicios. Sin embargo no todas las soluciones son de codigo abierto, algunas son de desarrollo propietario, en donde solo ofrecen el API al p'ublico pero no el c'odigo de la soluci'on como tal y manejan sus propias herramientas de virtualizaci'on.

\section{Alternativas de despliegue actuales}
En esta secci'on, se describir'an las herramientas mas significativas que existen indicando las ventajas y desventajas de cada una.

\subsection{Nixes}
Nixes es una herramienta usada para instalar, mantener, controlar y monitorear aplicaciones en PlanetLab \cite{Nixes13}. Nixes consiste de un conjunto de scripts bash, un archivo de configuracion, y un respositorio web, y puede automaticamente instalar, actualizar y resolver dependencias solo de paquetes RPM.\\
\\
Para sistemas de peque'na escala, Nixes es f'acil de usar: los usuarios simplement e crean el archivo de configuraci'on para cada aplicaci'on y modifican los scripts a desplegar en los nodos. Pero para grandes y complejos sistemas, Nixes no es efectivo, porque el no provee un mecanismo autom'atico de flujo de trabajo.

\subsection{SmartFrog}
SmartFrog (SF) es un framework para servicios de configuraci'on, descripci'on, despliegue y administraci'on del ciclo de vida. Consiste de un lenguaje declarativo, un motor que corre en los nodos remotos y ejecuta plantillas escritas en el lenguaje de SmartFrog y un modelo de componentes. El lenguaje soporta encapsulaci'on (que es similar a las clases de python), herencia y composici'on que permite personalizar y combinar configuraciones. SmartFrog, permite enlaces est'aticos y din'amicos entre componentes, que ayudan a soportar diferentes formas de conexi'on en tiempo de despliegue.\\
\\
El modelo de componentes SF, administra el ciclo de vida atraves de cinco estados: instalado, iniciado, terminado y fallido. Esto permite al motor del SmartFrog detectar fallas y reiniciar autom'aticamente re-despliegues de los componentes \cite{Smart09}.
\\
SmartFrog es desarrollado y mantenido por un equipo de investigaci'on en los laboratorios de Hewlett-Packard en Bristol, Inglaterra, asi como por el laboratorio Europeo de Hewlett-Packard y adicional con contribuciones de otros usuarios de SmartFrog y desarrolladores externos a HP. Se utiliza en la investigaci'on de HP especificamente en la automatizaci'on de la infraestructura y automatizaci'on de servicios, adem'as de ser utilizado en determinados productos de HP.

\subsection{Radia}
Herramienta de administraci'on de cambios que utiliza un enfoque basado en modelos \cite{Radia15}. Para cada dispositivo administrado, el administrador define un estado deseado, el cual es mantenido como un modelo en un repositorio central. Nixes, usa seis maquinascalculosmodelos: paquete (configuraci'on, instalaci'on, entradas de registro, binarios, entre otras); mejores pr'acticas; dependencias de software (relaciones con otros componentes de software, sistemas operativos y hardware); infraestructura (servidores, almacenamiento y elementos de red); inventarios de software (software instalado actualmente) e interoperabilidad entre modelos de servicios administrados. 

\subsection{Cobbler}
Cobbler es una plataforma que busca el r'apido despliegue de servidores y en general computadores en una infra-estructura de red, se basa en el modelo de scripts y cuenta con una completa base de simples comandos, que permite hacer despliegues de manera r'apida y con poca intervenci'on humana. Cobbler al igual que SmartFrog es capaz de instalar m'aquinas f'isicas y m'aquinas virtuales. Cobbler, es una peque'na y ligera aplicaci'on, que es extremadamente f'acil de usar para peque'nos o muy grandes despliegues. \cite{6}

\subsection{Amazon EC2}
Amazon EC2 es un API propietario de Amazon y maneja un enfoque manual, que permite desplegar im'agenes de m'aquinas virtuales conocidas como AMI (Amazon Machine Images) \cite{9}, que son las im'agenes que se utilizan en Amazon para arrancar instancias. El concepto de las amis es similar a las m'aquinas virtuales de otros sistemas. B'asicamente est'an compuestas de una serie ficheros de datos que conforman la imagen y luego un xml que especifica ciertos valores necesarios para que sea una imagen v'alida para Amazon que es el image.manifest.xml. 

\subsection{HP Utility Data Center}
HP Utility Data Cente (UDC) es un producto comercial, que se centra en la administraci'on automatizada de servidores de red, usando el concepto de $"$infraestructura programable $"$. Los elementos de hardware, como nodos de servidores, switches, firewalls y elementos de almacenamiento, son cableados en una infraestructura de configuraci'on. El software de administraci'on UDC permite configurar combinaciones de estos componentes en servidores virtuales usando cableados virtuales. \cite{15}

\subsection{Oracle VM Templates}
Oracle VM Templates, es un producto comercial de la empresa Oracle, cuyo objetivo es realizar despliegues rapidos de aplicaciones Oracle y no-Oracle, con base en imagenes de software pre-configuradas manualmente. Cuenta con una interfaz grafica que permite crear y administrar servidores virtuales con facilidad. \cite{14}

\subsection{Chef}
Chef es una herramienta de gesti'on de la configuraci'on escrito en Ruby y Erlang. Utiliza un lenguaje de dominio especifico escrito tamibi'en en Ruby para la escritura y configuraci'on de "recetas". Estas recetas contienen los recursos que deben ser creados. Chef se puede integrar con plataformas basadas en la nube, como Rackspace, Internap, Amazon EC2, Cloud Platform Google, OpenStack, SoftLayer y Microsoft Azure. Chef contiene soluciones para sistemas de peque~na y gran escala. \cite{Chef15}\\
\\
Es uno de los cuatro principales sistemas de gesti'on de la configuraci'on en Linux, junto con Cfengine, Bcfg2 y Puppet.

\subsection{Puppet}
Puppet es una herramienta dise~nada para administrar la configuraci'on de sistemas similares a Unix y a Microsoft Windows de forma declarativa. El usuario describe los recursos del sistema y sus estados utilizando el lenguaje declarativo que proporciona Puppet. Esta informaci'on es almacenada en archivos denominados manifiestos Puppet. Puppet descubre la informaci'on del sistema a trav'es de una utilidad llamada Facter, y compila los manifiestos en un cat'alogo espec'ifico del sistema que contiene los recursos y la dependencia de dichos recursos, estos cat'alogos son ejecutados en los sistemas de destino. \cite{Pupet15}

\subsection{Cfengine}
Cfengine es un sistema basado en el lenguaje escrito por Mark Burgess, dise~nado espec'ificamente para probar y configurar software. Cfengine es como un lenguaje de muy alto nivel. La idea de Cfengine es crear un 'unico archivo o conjunto de archivos de configuraci'on que describen la configuraci'on de cada host de la red. Cfengine se ejecuta en cada host, y analiza cada archivo (o archivos), que especifica una pol'itica para la configuraci'on del sistema; la configuraci'on del host es verificada contra el modelo y, si es necesario, cualquier desviaci'on de la configuraci'on es corregida. \cite{cfengine15}

\subsection{Bcfg2}
Bcfg2 est'a escrito en Python y permite gestionar la configuraci'on de un gran n'umero de ordenadores mediante un modelo de configuraci'on central. Bcfg2 funciona con un modelo simple de configuraci'on del sistema, modelando elementos intuitivos como paquetes, servicios y archivos de configuraci'on (as'i como las dependencias entre ellos). Este modelo de configuraci'on del sistema se utiliza para la verificaci'on  y validaci'on, permitiendo una auditor'ia robusta de los sistemas desplegados. La especificaci'on de la configuraci'on de Bcfg2 está escrita utilizando un modelo XML declarativo. Toda la especificaci'on puede ser validada utilizando los validadores de esquema XML ampliamente disponibles. \cite{bdfg215}

\subsection{vagrant}

\subsection{ansible}



\section{Dise'no}
\chapter{Autenticaci'on y Protecci'on}
\label{capauthentication}

\section{Autenticaci'on}
La autenticaci'on es el proceso de demostrar la identidad al sistema. La identidad es un factor importante en las decisiones de control de acceso. Las solicitudes se conceden o deniegan en parte sobre la base de la identidad del solicitante.\\
\\
El VirtShell, el API REST utiliza un esquema HTTP personalizado basado en una llave-HMAC (Hash Message Authentication Code) para la autenticaci'on. Para autenticar una solicitud, primero se concatenan los elementos seleccionados de la solicitud para formar una cadena. A continuaci'on, utiliza una clave secreta de acceso para calcular el HMAC de esa cadena. Informalmente, se le denomina a este proceso \"la firma de la solicitud\", y se denomina a la salida del algoritmo HMAC la "firma", ya que simula las propiedades de seguridad de una firma real. Por 'ultimo, se agrega esta firma como un par'ametro de la petici'on, con la sintaxis descrita en esta secci'on.\\
\\
Cuando el sistema recibe una solicitud fehaciente, se obtiene la clave secreta de acceso que dicen tener, y lo utiliza de la misma manera que se calcula una "firma" del mensaje que recibi'o. A continuaci'on, compara la firma que se calcula con la firma presentada por el solicitante. Si las dos firmas coinciden, el sistema llega a la conclusi'on de que el solicitante debe tener acceso a la clave secreta de acceso, y por lo tanto act'ua con la autoridad del principal al que se emiti'o la clave. Si las dos firmas no coinciden, la solicitud se descarta y el sistema responde con un mensaje de error.\\
\\
Ejemplo de una petici'on autenticada:

\medskip
\begin{lstlisting}
  GET /api/virtshell/packages/{packageId} HTTP/1.1
  Host: host1.edu.co
  Date: Fri, 01 Jul 2011 19:37:58 +0000

  Authorization: 0PN5J17HBGZHT7JJ3X82:frJIUN8DYpKDtOLCwo//yllqDzg= 
\end{lstlisting}

\subsection{Authentication Header}

El API REST utiliza el encabezado de autorizaci'on HTTP para pasar informaci'on de autenticaci'on. Bajo el esquema de autenticaci'on de VirtShell, el encabezado de autorizaci'on tiene la siguiente forma.

\medskip
\begin{lstlisting}
  Authorization: UserId:Signature
\end{lstlisting}
\medskip

Los usuarios tendr'an un ID de clave de acceso (VirtShell Access Key ID) y una clave secreta de acceso (VirtShell Secret Access Key) cuando se registran. Para la petici'on de autenticaci'on, el elemento de VirtShell Access Key Id identifica la clave secreta que se utiliz'o para calcular la firma, y (indirectamente) el usuario que realiza la solicitud.\\
\\
Para la firma de los elementos de la petici'on se usa el RFC 2104HMAC-SHA1, por lo que la parte de la firma de la cabecera autorizaci'on variar'a de una petici'on a otra. Si la solicitud de la firma calculada por el sistema coincide con la firma incluida en la solicitud, el solicitante habr'a demostrado la posesi'on de la clave secreta de acceso. La solicitud ser'a procesada bajo la identidad, y con la autoridad, de la promotora que se emiti'o la clave.\\
\\
A continuaci'on se muestra la pseudo-gram'atica que ilustra la construcci'on de la cabecera de la solicitud de autorizaci'on (
\textbackslash{}n significa el punto de c'odigo Unicode U +000 A com'unmente llamado salto de l'inea).

\medskip
\begin{lstlisting}[basicstyle=\tiny]
  Authorization = VirtShellUserId + ":" + Signature;

  Signature = Base64( HMAC-SHA1( UTF-8-Encoding-Of( YourSecretAccessKeyID, StringToSign ) ) );

  StringToSign = HTTP-Verb + "\n" +
  Host + "\n" +
  Content-MD5 + "\n" +
  Content-Type + "\n" +
  Date + "\n" +
  CanonicalizedResource;

  CanonicalizedResource = <HTTP-Request-URI, from the protocol name up to the query string (resource path)>
\end{lstlisting}

HMAC-SHA1 es un algoritmo definido por la RFC 2104 (ver la RFC 2104 con llave Hashing para la autenticaci'on de mensajes).\\
\\
El algoritmo toma como entrada dos cadenas de bytes: una clave y un mensaje. Para la solicitud de autenticaci'on, se utiliza la clave secreta (YourSecretAccessKeyID) como la clave, y la codificaci'on UTF-8 del StringToSign como el mensaje. La salida de HMAC-SHA1 es tambi'en una cadena de bytes, llamado el resumen. El par'ametro de la petici'on de la Firma se construye codificada en Base64.

\subsection{Solicitud can'onica para firmar}

Cuando el sistema recibe una solicitud autenticada, compara la solicitud de firma calculada con la firma proporcionada en la solicitud de StringToSign. Por esta raz'on, se debe calcular la firma con el mismo m'etodo utilizado por VirtShell. A este proceso de poner una solicitud en una forma acordada para la firma se denomino "canonizaci'on".

\subsection{Tiempo de sello}

Un sello de tiempo v'alido (utilizando el HTTP header Date) es obligatorio para solicitudes autenticadas. Por otra parte, el tiempo del sello enviado por un usuario que se encuentra incluido en una solicitud autenticada debe estar dentro de los 15 minutos de la hora del sistema cuando se recibe la solicitud. En caso contrario, la solicitud fallar'a con el c'odigo de estado de error RequestTimeTooSkewed. La intenci'on de estas restricciones es limitar la posibilidad de que solicitudes interceptadas pueden ser reproducidos por un adversario.Para una mayor protecci'on contra las escuchas, se debe utilizar el transporte HTTPS para solicitudes autenticadas.

\subsection{Ejemplos de autenticaci'on}

\scriptsize
\begin{tabular}{|l|l|} \hline
\textbf{Parametro} & \textbf{Valor} \\ \hline
VirtShellUserId  & 13010f3e-3f46-4889-b989-592ce8fb30c6 \\ \hline
\multicolumn{1}{|m{3.5cm}|}{VirtShellSecretAccessKey} & 
\multicolumn{1}{m{9cm}|} {
                            \raggedright c991f519-bed0-4dab-9165-6d3f722dc845 \\
                            \textbf{Base64:} \\ Yzk5MWY1MTktYmVkMC00ZGFiLTkxNjUtNmQ5ZjcyMmRjODQ1
                          } \tabularnewline \hline
\end{tabular}
\normalsize

\subsubsection{Ejemplo de un objeto con GET}

Este es un ejemplo que consulta por un host dado su identificador. \\
\\
\vspace{1cm}
\scriptsize
\begin{tabular}{|l|l|} \hline
\textbf{Request} & \textbf{StringToSign} \\ \hline
\multicolumn{1}{|m{7.5cm}|}{
      \raggedright GET /api/virtshell/hosts/5713b48a-8d73-11e5-8994-feff819cdc9f HTTP/1.1 \\
      Host: host1.edu.co \\
      Date: Tue, 27 Mar 2007 19:36:42 +0000 \\
      Authorization: 13010f3e-3f46-4889-b989-592ce8fb30c6: \\
      Yzk5MWY1MmVkMC00ZGFiLTtNmQ5ZjcyMmRjODQ1 } & 
\multicolumn{1}{m{8cm}|}{
      \raggedright GET\textbackslash{}n \\
      host1.edu.co\textbackslash{}n \\
      \textbackslash{}n \\
      \textbackslash{}n \\
      Tue, 27 Mar 2007 19:36:42 +0000\textbackslash{}n \\ 
      /api/virtshell/hosts/5713b48a-8d73-11e5-8994-feff819cdc9f} \tabularnewline \hline
\end{tabular}
\normalsize

\subsubsection{Ejemplo de un objeto con DELETE}

Este ejemplo remueve un usuario.\\
\\
\vspace{1cm}
\scriptsize
\begin{tabular}{|l|l|} \hline
\textbf{Request} & \textbf{StringToSign} \\ \hline
\multicolumn{1}{|m{7.4cm}|}%
{\raggedright DELETE /api/virtshell/users/5fd13cc8-8d73-11e5-8994-feff819cdc9f HTTP/1.1 \\
 Host: host1.edu.co \\
 Date: Tue, 27 Mar 2007 21:20:27 +0000 \\
 Authorization: 13010f3e-3f46-4889-b989-592ce8fb30c6: Yzk5MWY1MmVkMC00ZGFiLTtNmQ5ZjcyMmRjODQ1 } & \multicolumn{1}{m{8cm}|}%
{\raggedright DELETE\textbackslash{}n \\
 host1.edu.co\textbackslash{}n \\
 \textbackslash{}n \\
 \textbackslash{}n \\
 Tue, 27 Mar 2007 21:20:27 +0000\textbackslash{}n \\ /api/virtshell/users/5fd13cc8-8d73-11e5-8994-feff819cdc9f} \tabularnewline \hline
\end{tabular}

\normalsize
\section{Protecci'on}


\chapter{API}
\label{capapi}

\section{Que es un API?}
API significa ``Application Programming Interface", y como término, especifica cómo debe interactuar el software.\\
\\
En términos generales, cuando nos referimos a las API de hoy, nos referimos más concretamente a las API web, que son manejadas a través del protocolo de transferencia de hipertexto (HTTP). Para este caso específico, entonces, una API especifica cómo un consumidor puede consumir el servicio que el API expone: cuales URI están disponibles, qué métodos HTTP puede utilizarse con cada URI, que parámetros de consulta se acepta, lo que los datos que pueden ser enviados en el cuerpo de la petición, y lo que el consumidor puede esperar como respuesta.

\section{Tipos de APIs}
Las APIs web pueden ser divididas en dos categorías generales:
\begin{itemize}
\item Remote Procedure Call (RPC)
\item REpresentational State Transfer (REST)
\end{itemize}

\subsection{RPC}
RPC se caracteriza generalmente como un único URI a través del protocolo HTTP en el que se pueden llamar muchas operaciones, por lo general solo se usan las operaciones GET y POST. Cuando se pasa una solicitud estructurada, esta incluye el nombre de la operación a invocar y los argumentos que desea pasar a la operación; la respuesta será devuelta tambien en un formato estructurado.\\
\\
Una cosa a tener en cuenta es que por lo general RPC hace todo el informe de errores en el cuerpo de la respuesta; el código de estado HTTP no variará, lo que significa que hay que fijarse en el valor de retorno para determinar si se ha producido un error.\\
\\
Muchas implementaciones de RPC también proporcionan documentación para sus usuarios finales a través del protocolo en sí. Por ejemplo para SOAP lo hace a traves del WSDL. Esta característica de autodocumentado puede proporcionar información muy valiosa para el consumidor sobre cómo interactuar con el servicio.\\
\\
En resumen, los puntos a tener en cuenta acerca de RPC son:

\begin{itemize}
\item Un extremo de servicio, muchas operaciones.
\item Un extremo de servicio, un método HTTP (normalmente POST).
\item Formato de solicitud predecible estructurada, formato de respuesta estructurada y predecible.
\item Formato de informe de errores predecibles estructurada.
\item Documentación estructurada de operaciones disponibles.
\end{itemize}

Dicho todo esto, RPC es a menudo un mal elección para las API web:\\
\\
\begin{itemize}
\item No se puede determinar a través de la URI la disponibilidad de varios recursos.
\item Falta de almacenamiento en caché de HTTP
\item La imposibilidad de usar verbos HTTP nativos para operaciones comunes 
\item Falta de códigos de respuesta, se requiere la introspección de los resultados para determinar si se ha producido un error.
\item Los clientes no podran consumir formatos de serialización alternativos.
\item Los formatos de mensajes a menudo imponen restricciones innecesarias sobre los tipos de datos que se pueden enviar o devolver.
\end{itemize}

En pocas palabras, RPC no utiliza las capacidades completas del protocolo HTTP.

\subsection{REST}
REST es un estilo de arquitectura de software que proporciona un enfoque pr'actico y consistente para solicitar y modificar datos en torno a la especificación del protocolo HTTP. \\
\\
El termino REST es la abreviatura para ``Representational State Transfer.", el cual aprovecha las fortalezas de los protocolos HTTP y HTTPS. Un buen API REST debe constar de:
\begin{itemize}
\item Usar URIs como identificadores únicos de los recursos.
\item Aprovechar el espectro completo de verbos HTTP para las operaciones sobre los recursos.
\item Permitir el manejo de varios formatos de representacion de recursos.
\item Proporcionar vinculación entre los recursos para indicar las relaciones. (Por ejemplo, enlaces hipermedia, como los encontrados en los antiguos documentos HTML plano)
\end{itemize}

Todo esta teoría dice cómo deben actuar los servicios REST, pero dice muy poco acerca de la forma de implementarlos. REST es más una consideración arquitectónica. Sin embargo, esto significa que al momento de diseñar un API REST se debe considerar muchas opciones, algunas como que formato se va a exponer, como se reportaran los errores, como se comunicara los metodos HTTP disponibles, como se manejaran caracteristicas de autenticacion, como se suministraran las credenciales en cada peticion, etc.\\
\\
En pocas palabras, la mayoría un API REST proporciona una increíble flexibilidad y potencia, pero requiere de tomar muchas decisiones con el fin de proporcionar una sólida experiencia y calidad para los consumidores.

\subsection{VirtShell API REST}
En el VirtShell API REST un usuario env'ia una solicitud al servidor para realizar una acci'on determinada (como la creaci'on, recuperaci'on, actualizaci'on o eliminaci'on de un recurso virtual), y el servidor realiza la acci'on y env'ia una respuesta, a menudo en la forma de una representaci'on del recurso especificado.\\
\\
En el VirtShell API, el usuario especifica una acci'on con un verbo HTTP como POST, GET, PUT o DELETE. Especificando un recurso por un URI 'unico global de la siguiente forma: \\
\\
https://[host]:[port]/virtshell/api/v1/resourcePath?parameters\\
\\
Debido a que todos los recursos del API tienen una 'unica URI HTTP accesible, REST permite el almacenamiento en cach'e de datos y est'a optimizado para trabajar con una infraestructura distribuida de la web.\\
\\
En esta sección se detalla los recursos y operaciones que puede realizar un usuario del API para realizar aprovisionamientos autom'aticos desde cualquier plataforma de desarrollo. El VirtShell API provee accesso a los objetos en el VirtShell Server, esto incluye los hosts, imagenes, archivos, templates, aprovisionadores, instancias, grupos y usuarios. Por medio del API podr'a crear ambientes, m'aquinas virtuales y contenedores personalizados, realizar configuraciones y administrar los recursos f'isicos y virtuales de manera program'atica. \\

\section{Formato de entrada y salida}
JSON (JavaScript Object Notation) es un formato de datos com'un, independiente del lenguaje que proporciona una representaci'on de texto simple de estructuras de datos arbitrarias. Para obtener m'as informaci'on, ver json.org.\\
\\
El VirtShell API solo soporta el formato json para intercambio de informaci'on. Cualquier solicitud que no se encuentre en formato json resultara en un error con codigo 406 (Content Not Acceptable Error).

\section{Codigos de error}
Aqui se presenta una lista de codigos de error que pueden resultar de una petici'on al API en cualquier recurso.

\begin{itemize}
\item \textbf{400 Bad Request} La solicitud no pudo ser procesada con 'exito porque el URI no era v'alido. El cuerpo de la respuesta contendr'a una raz'on del fracaso de la petici'on. Esta respuesta indica error permanente.

\item \textbf{403 Forbidden} La solicitud no pudo ser procesada con 'exito porque la identidad del usuario no tiene acceso suficiente para procesar la solicitud. Esta respuesta indica error permanente.

\item \textbf{406 Content Not Acceptable} Un recurso genera este error de acuerdo al tipo de cabeceras enviadas en la petici'on. Esta respuesta indica un error permanete e indica un formato de salida no soportado. La respuesta de este tipo de error no contiene un contenido debido a la inhabilidad del servidor para generar una respuesta en el formato solicitado.

\item \textbf{404 Not Found} La solicitud no pudo ser procesada con 'exito porque la solicitud no era v'alida. Lo m'as probable es que no se encontró la url. Esta respuesta indica error permanente.

\item \textbf{500 Server Error} La solicitud no pudo ser procesada debido a que el servidor encontr'o una condici'on inesperada que le impidi'o cumplir con la petici'on.

\item \textbf{501 Not Implemented} La solicitud no se pudo completar porque el servidor o bien no reconoce el m'etodo de petici'on o el recurso solicitado no existe.

\end{itemize}

Los errores que no sean de codigo 406 (Content Not Acceptable) contienen una respuesta en formato json, que contiene un breve mensaje explicado el error con m'as detalle. Por ejemplo, una consulta POST /virtshell/api/v1/hosts, con un cuerpo vacio, dar'ia lugar a la siguiente respuesta:

\vspace{1cm}
\begin{lstlisting}[style=json]
HTTP/1.1 400 Bad Request
Content-Type: application/json

{"error": "Missing input for create instance"}
\end{lstlisting}

\section{API Resources}

\subsection{Hosts}
Representan las m'aquinas f'sicas; un host es un anfitrion de maquinas virtuales o contenedores. Los m'etodos soportados son:

\begin{center}
 \begin{tabular}{| l | l | l | l |}
 \hline
  \rowcolor{blueapi}
  \textbf{Acci'on} & \textbf{Metodo HTTP} & \textbf{Solicitud HTTP} & \textbf{Descripci'on} \\ [0.5ex] 
  \hline\hline
  get & GET & /hosts/id & Gets one host by ID. \\
  \hline
  list & GET & /hosts & Retrieves the list of hosts. \\
  \hline  
  create & POST & /hosts/ & Inserts a new host configuration. \\
  \hline
  delete & DELETE & /hosts/id & Deletes an existing host. \\
  \hline  
  update & PUT & /hosts/id & Updates an existing host. \\ [1ex] 
  \hline
\end{tabular}
\end{center}

Representaci'on del recurso de un host:

\medskip
\begin{lstlisting}[style=json]
{
  "uuid": string,
  "name": string,
  "os": string,
  "memory": string,
  "capacity": string,
  "enabled": string,
  "type":string,
  "local_ipv4": string,
  "local_ipv6": string,
  "public_ipv4": string,
  "public_ipv6": string,
  "instances": [ instance_resource],
  "created":["at": number, "by": number]
}
\end{lstlisting}

Ejemplo:

\medskip
\begin{lstlisting}[style=json]
{
  "uuid": "ab8076c0-db91-11e2-82ce-0002a5d5c51b",
  "name": "host-01-pdn",
  "os": "Ubuntu_12.04_3.5.0-23.x86_64",
  "memory": "16GB",
  "capacity": "120GB",
  "enabled": "true|false",
  "type":"StorageOptimized|GeneralPurpose|HighPerformance",
  "local_ipv4": "15.54.88.19",
  "local_ipv6": "ff06:0:0:0:0:0:0:c3",
  "public_ipv4": "10.54.88.19",
  "public_ipv6": "yt06:0:0:0:0:0:0:c3",
  "instances": [
    ... instances resource is here
  ],
  "created":["at":"timestamp", "by":1234]
}
\end{lstlisting}

\subsubsection{Ejemplos de peticiones HTTP}

\paragraph{Crear un nuevo host - POST /api/virtshell/v1/hosts} ~\\

\begin{lstlisting}[style=json]
curl -sv -X POST \
  -H 'accept: application/json' \
    -H 'X-VirtShell-Authorization: UserId:Signature' \
  -d '{"name": "host-01-pdn",
       "os": "Ubuntu_12.04_3.5.0-23.x86_64",
       "memory": "16GB",
       "capacity": "120GB",
       "enabled": "true",
       "type" : "GeneralPurpose",
       "local_ipv4": "15.54.88.19",
         "local_ipv6": "ff06:0:0:0:0:0:0:c3",
       "public_ipv4": "10.54.88.19",
       "public_ipv6": "yt06:0:0:0:0:0:0:c3"}' \
   'http://localhost:8080/virtshell/api/v1/hosts'
\end{lstlisting}

Response:

\begin{lstlisting}[style=json]
HTTP/1.1 200 OK
Content-Type: application/json
{ "create": "success" }
\end{lstlisting}

\paragraph{Obtener un host- GET /api/virtshell/v1/hosts/:id} ~\\

\begin{lstlisting}[style=json]
curl -sv -H 'accept: application/json' 
     -H 'X-VirtShell-Authorization: UserId:Signature' \ 
     'http://localhost:8080/api/virtshell/v1/hosts?id=ab8076c0-db91-11e2-82ce-0002a5d5c51b'
\end{lstlisting}

Response:

\begin{lstlisting}[style=json]
HTTP/1.1 200 OK
Content-Type: application/json
{
  "uuid": "ab8076c0-db91-11e2-82ce-0002a5d5c51b",
  "name": "host-01-pdn",
  "os": "Ubuntu_12.04_3.5.0-23.x86_64",
  "memory": "16GB",
  "capacity": "120GB",
  "enabled": "true",
  "type" : "StorageOptimized",
  "local_ipv4": "15.54.88.19",
  "local_ipv6": "ff06:0:0:0:0:0:0:c3",
  "public_ipv4": "10.54.88.19",
  "public_ipv6": "yt06:0:0:0:0:0:0:c3",
  "instances": [
    {
      "name": "name1",
      "id": "72C05559-0590-4DA6-BE56-28AB36CB669C"
    },
    {
      "name": "name2",
      "id": "17173587-C4E9-4369-9C43-FCBF5E075973"
    }
  ],
  "created":["at":"20130625105211", "by":10]
}
\end{lstlisting}

\paragraph{Obtener todos los host - GET /api/virtshell/v1/hosts} ~\\

\begin{lstlisting}[style=json]
curl -sv -H 'accept: application/json' 
     -H 'X-VirtShell-Authorization: UserId:Signature' \ 
     'http://localhost:8080/api/virtshell/v1/hosts'
\end{lstlisting}

Response:

\begin{lstlisting}[style=json]
HTTP/1.1 200 OK
Content-Type: application/json
{
  "hosts": [
    {
      "uuid": "ab8076c0-db91-11e2-82ce-0002a5d5c51b",
      "name": "host-01-pdn",
      "os": "Ubuntu_12.04_3.5.0-23.x86_64",
      "memory": "16GB",
      "capacity": "120GB",
      "enabled": "true",
      "type" : "StorageOptimized",
      "local_ipv4": "15.54.88.19",
      "local_ipv6": "ff06:0:0:0:0:0:0:c3",
      "public_ipv4": "10.54.88.19",
      "public_ipv6": "yt06:0:0:0:0:0:0:c3",
      "instances": [
        {
          "name": "name1",
          "id": "72C05559-0590-4DA6-BE56-28AB36CB669C"
        },
        {
          "name": "name2",
          "id": "17173587-C4E9-4369-9C43-FCBF5E075973"
        }
      ],
      "created":["at":"20130625105211", "by":10]
    },
    {
      "uuid": "ab8076c0-db91-11e2-82ce-0002a5d5c51b",
      "name": "host-01-pdn",
      "os": "Ubuntu_12.04_3.5.0-23.x86_64",
      "memory": "16GB",
      "capacity": "120GB",
      "enabled": "true",
      "type" : "GeneralPurpose",
      "local_ipv4": "15.54.88.19",
      "local_ipv6": "ff06:0:0:0:0:0:0:c3",
      "public_ipv4": "10.54.88.19",
      "public_ipv6": "yt06:0:0:0:0:0:0:c3",
      "instances": [
        {
          "name": "name3",
          "id": "DE11CC9A-482F-4033-A7F8-503EE449DD0A"
        },
        {
          "name": "name4",
          "id": "17173587-C4E9-4369-9C43-FCBF5E075973"
        },    
      ],
      "created":["at":"20130625105211", "by":10]
    }
  ]
}   
\end{lstlisting}

\paragraph{Actualizar un host - PUT /api/virtshell/v1/hosts/:id} ~\\

\begin{lstlisting}[style=json]
curl -sv -X PUT \
  -H 'accept: application/json' \
    -H 'X-VirtShell-Authorization: UserId:Signature' \
  -d '{"memory": "24GB",
     "capacity": "750GB"}' \
   'http://localhost:8080/api/virtshell/v1/hosts?id=ab8076c0-db91-11e2-82ce-0002a5d5c51b'
\end{lstlisting}

Response:

\begin{lstlisting}[style=json]
HTTP/1.1 200 OK
Content-Type: application/json

{ "update": "success" }
\end{lstlisting}

\paragraph{Eliminar un host - DELETE /api/virtshell/v1/hosts/:id} ~\\

\begin{lstlisting}[style=json]
curl -sv -X DELETE \
   -H 'accept: application/json' \
   -H 'X-VirtShell-Authorization: UserId:Signature' \
   'http://localhost:8080/api/virtshell/v1/hosts?id=ab8076c0-db91-11e2-82ce-0002a5d5c51b'
\end{lstlisting}

Response:

\begin{lstlisting}[style=json]
HTTP/1.1 200 OK
Content-Type: application/json
```
```json
{ "delete": "success" }
\end{lstlisting}
\section{Files}
Representan toda clase de archivos que se requieran para crear o aprovisionar m'aquinas virtuales o contenedores. Los metodos soportados son:

\begin{center}
 \begin{tabular}{| l | l | l | l |}
 \hline
  \rowcolor{blueapi}
  \textbf{Acci'on} & \textbf{Metodo HTTP} & \textbf{Solicitud HTTP} & \textbf{Descripci'on} \\ [0.5ex] 
  \hline\hline
  get & GET & /files/id & Gets one file by ID. \\
  \hline
  create & POST & /files/ & upload a new file. \\
  \hline
  delete & DELETE & /files/id & Deletes an existing file. \\
  \hline  
  update & PUT & /images/id & Updates an existing file. \\ [1ex]  
  \hline
\end{tabular}
\end{center}

\vspace{1cm}
Representaci'on del recurso de un archivo:
\vspace{1cm}

\begin{lstlisting}[style=json]
{
  "uuid": "ab8076c0-db91-11e2-82ce-0002a5d5c51b",
  "name": "file_name.extension",
  "folder_name" : "folder_name",
  "download_url": "https://<host>:<port>/api/virtshell/v1/files/folder_name/file.txt",
  "created":["at":"timestamp", "by":user_id]
}
\end{lstlisting}

Ejemplo:

\medskip
\begin{lstlisting}[style=json]
{
  "uuid": "ab8076c0-db91-11e2-82ce-0002a5d5c51b",
  "name": "ubuntu_seed_14-04.tex",
  "folder_name" : "ubuntu_seeds",
  "download_url": "https://<host>:<port>/api/virtshell/v1/files/ubuntu_seeds/ubuntu_seed_14-04.tex",
  "created": ["at":"20130625105211", "by":10]
}
\end{lstlisting}

\subsection{Ejemplos de peticiones HTTP}

\subsubsection{Subir un nuevo archivo - POST /virtshell/api/v1/images}

\begin{lstlisting}[style=json]
curl -X POST \
  -H 'accept: application/json' \
  -H 'X-VirtShell-Authorization: UserId:Signature' \
  -H "Content-Type: multipart/form-data" \
  -F "file_data=@/path/to/file/seed_file.txt;filename=seed_file_ubuntu-14_04.txt" \
  -F "folder_name=ubuntu_seeds" \
  'http://<host>:<port>/api/virtshell/v1/files'
\end{lstlisting}

\vspace{1cm}
Respuesta:
\vspace{1cm}

\begin{lstlisting}[style=json]
HTTP/1.1 200 OK
Content-Type: application/json
{ 
  "create": "success",
  "location": "http://<host>:<port>/api/virtshell/v1/files/ubuntu_seeds/seed_file_ubuntu-14_04.txt" 
}
\end{lstlisting}

\subsubsection{Obtener un archivo - GET /virtshell/api/v1/files/:id}

Para descargar un archivo, primero recibira la url apropiada que viene en la metadata provista por la url. Luego podra descargarlo usando la url.

\begin{lstlisting}[style=json]
curl -sv -H 'accept: application/json' 
     -H 'X-VirtShell-Authorization: UserId:Signature' \ 
     'http://<host>:<port>/api/virtshell/v1/files/?id=ab8076c0-db91-11e2-82ce-0002a5d5c51b'
\end{lstlisting}

\vspace{1cm}
Respuesta:
\vspace{1cm}

\begin{lstlisting}[style=json]
HTTP/1.1 200 OK
Content-Type: application/json
{
  "uuid": "ab8076c0-db91-11e2-82ce-0002a5d5c51b",
  "name": "file_name.extension",
  "folder_name" : "folder_name",
  "download_url": "http://<host>:<port>/api/virtshell/v1/files/ubuntu_seeds/seed_file_ubuntu-14_04.txt",
  "created":["at":"timestamp", "by":user_id] 
}
\end{lstlisting}

\subsubsection{Actualizar un archivo - PUT /virtshell/api/v1/files/:id}

\begin{lstlisting}[style=json]
curl -sv -X PUT \
  -H 'accept: application/json' \
  -H 'X-VirtShell-Authorization: UserId:Signature' \
  -H "Content-Type: multipart/form-data" \
  -F "file_data=@/path/to/file/seed_file.txt;filename=seed_file_ubuntu-14_04_v2.txt" \
   'http://localhost:8080/api/virtshell/v1/file?id=8de7b824-d7d1-4265-a3a6-5b46cc9b8ed5'
\end{lstlisting}

\vspace{1cm}
Respuesta:
\vspace{1cm}

\begin{lstlisting}[style=json]
HTTP/1.1 200 OK
Content-Type: application/json

{ "update": "success" }
\end{lstlisting}


\subsubsection{Eliminar un archivo - DELETE /virtshell/api/v1/files/:id}

\begin{lstlisting}[style=json]
curl -sv -X DELETE \
   -H 'accept: application/json' \
   -H 'X-VirtShell-Authorization: UserId:Signature' \
   'http://localhost:8080/api/virtshell/v1/fles?id=ab8076c0-db91-11e2-82ce-0002a5d5c51b'
\end{lstlisting}

\vspace{1cm}
Respuesta:
\vspace{1cm}

\begin{lstlisting}[style=json]
HTTP/1.1 200 OK
Content-Type: application/json
```
```json
{ "delete": "success" }
\end{lstlisting}

\subsection{Images}
Representan imagenes de m'aquinas virtuales o contenedores. Los metodos soportados son:

\begin{center}
 \begin{tabular}{| l | l | l | l |}
 \hline
  \rowcolor{blueapi}
  \textbf{Acci'on} & \textbf{Metodo HTTP} & \textbf{Solicitud HTTP} & \textbf{Descripci'on} \\ [0.5ex] 
  \hline\hline
  get & GET & /images/id & Gets one image by ID. \\
  \hline
  list & GET & /images & Retrieves the list of images. \\
  \hline  
  create & POST & /images/ & Inserts a new image. \\
  \hline
  delete & DELETE & /images/id & Deletes an existing image. \\ [1ex] 
  \hline
\end{tabular}
\end{center}

\vspace{1cm}
Representaci'on del recurso de una imagen:
\vspace{1cm}

\begin{lstlisting}[style=json]
{
  "id": string,
  "name": string,
  "type": string,
  "os": string, 
  "release": string,
  "version": string, 
  "variant": string, 
  "arch": string, 
  "timezone": "America/Bogota", 
  "key": string,
  "preseed_url": url,
  "download_url": url,
  "permissions" : string,
  "created":["at": timestamp,"by": string],
  "details": string
}
\end{lstlisting}

Ejemplo:

\medskip
\begin{lstlisting}[style=json]
{
  "id": "kj5436c0-dc94-13tg-82ce-9992b5d5c51b",
  "name": "ubuntu_server_14.04.2_amd64",
  "type": "iso",
  "os": "ubuntu", 
  "release": "trusty",
  "version": "14.04.2", 
  "variant": "server", 
  "arch": "amd64", 
  "timezone": "America/Bogota", 
  "preseed_url": "https://<host>:<port>/api/virtshell/v1/files/seeds/seed_ubuntu14-04.txt",
  "permissions" : "rwxrw----",
  "created":["at":"20150625105211","by":10]
}
\end{lstlisting}

\subsubsection{Ejemplos de peticiones HTTP}

\paragraph{Crear una nueva imagen - POST /virtshell/api/v1/images} ~\\

\begin{lstlisting}[style=json]
curl -sv -X PUT \
  -H 'accept: application/json' \
  -H "Content-Type: text/plain" \
  -H 'X-VirtShell-Authorization: UserId:Signature' \
  -d '{"name": "ubuntu_server_14.04.2_amd64",
     "type": "iso",
     "os": "ubuntu", 
     "release": "trusty",
     "version": "14.04.2", 
     "variant": "server", 
     "arch": "amd64", 
     "timezone": "America/Bogota", 
     "key": "/home/callanor/.ssh/id_rsa.pub",
     "permissions" : "rwxrwxr--",
     "preseed_url": "https://<host>:<port>/api/virtshell/v1/files/seeds/seed_ubuntu14-04.txt"}' \
   'http://localhost:8080/api/virtshell/v1/image'
\end{lstlisting}

\vspace{1cm}
Respuesta:
\vspace{1cm}

\begin{lstlisting}[style=json]
HTTP/1.1 201 OK
Content-Type: application/json
{ "create": "success" }
\end{lstlisting}

\paragraph{Obtener una imagen - GET /virtshell/api/v1/images/:id} ~\\

\begin{lstlisting}[style=json]
curl -sv -H 'accept: application/json' 
     -H 'X-VirtShell-Authorization: UserId:Signature' \ 
     'http://localhost:8080/api/virtshell/v1/images?id=ab8076c0-db91-11e2-82ce-0002a5d5c51b'
\end{lstlisting}

\vspace{1cm}
Respuesta:
\vspace{1cm}

\begin{lstlisting}[style=json]
HTTP/1.1 200 OK
Content-Type: application/json
{
  "id": "kj5436c0-dc94-13tg-82ce-9992b5d5c51b",
  "name": "ubuntu_server_14.04.2_amd64",
  "type": "iso",
  "os": "ubuntu", 
  "release": "trusty",
  "version": "14.04.2", 
  "variant": "server", 
  "arch": "amd64", 
  "timezone": "America/Bogota", 
  "preseed_url": "https://<host>:<port>/api/virtshell/v1/files/seeds/seed_ubuntu_14_04.txt",
  "permissions" : "rwxrwxrwx",
  "created":["at":"20130625105211","by":10]
}
\end{lstlisting}

\paragraph{Obtener todas las imagenes - GET /virtshell/api/v1/images} ~\\

\begin{lstlisting}[style=json]
curl -sv -H 'accept: application/json' 
     -H 'X-VirtShell-Authorization: UserId:Signature' \ 
     'http://localhost:8080/api/virtshell/v1/images'
\end{lstlisting}

\vspace{1cm}
Respuesta:
\vspace{1cm}

\begin{lstlisting}[style=json]
HTTP/1.1 200 OK
Content-Type: application/json
{
  "images": [
    {
      "id": "b180ef2c-e798-4a8f-b23f-aaac2fb8f7e8",
      "name": "ubuntu_server_14.04.2_amd64",
      "type": "iso",
      "os": "ubuntu", 
      "release": "trusty",
      "version": "14.04.2", 
      "variant": "server", 
      "arch": "amd64", 
      "timezone": "America/Bogota", 
      "preseed_file": "https://<host>:<port>/api/virtshell/v1/files/seeds/seed_file.txt",
      "permissions" : "rwxrw----",
      "created":["at":"20130625105211","by":10]
    },
    {
      "id": "ca326181-bc84-4edb-bfc5-843037e7195e",
      "name": "centos_server",
      "type": "container",
      "os": "centos", 
      "version": "7", 
      "arch": "x86_64", 
      "download_url": "https://<host>:<port>/api/virtshell/v1/files/images/3514296#file-lxc-centos",
      "permissions" : "rwxrwxr--",
      "created":["at":"20140625105211","by":12]
    }
  ]
}  
\end{lstlisting}

\paragraph{Eliminar una imagen - DELETE /virtshell/api/v1/images/:id} ~\\

\begin{lstlisting}[style=json]
curl -sv -X DELETE \
   -H 'accept: application/json' \
   -H 'X-VirtShell-Authorization: UserId:Signature' \
   'http://<host>:<port>/api/virtshell/v1/images?id=ab8076c0-db91-11e2-82ce-0002a5d5c51b'
\end{lstlisting}

\vspace{1cm}
Respuesta:
\vspace{1cm}

\begin{lstlisting}[style=json]
HTTP/1.1 200 OK
Content-Type: application/json
```
```json
{ "delete": "success" }
\end{lstlisting}

\subsection{Packages}
Representan paquetes de software que se ejecutan en las m'aquinas virtuales o contenedores. Los metodos soportados son:

\begin{center}
 \begin{tabular}{| l | l | l | l |}
 \hline
  \rowcolor{blueapi}
  \textbf{Acci'on} & \textbf{Metodo HTTP} & \textbf{Solicitud HTTP} & \textbf{Descripci'on} \\ [0.5ex] 
  \hline\hline
  install & POST & /install\_packages/ & Install one or more packages. \\
  \hline
  upgrade & POST & /upgrade\_packages/ & Upgrade one or more packages. \\
  \hline
  remove & POST & /remove\_packages/ & Remove one or more packages. \\ [1ex] 
  \hline
\end{tabular}
\end{center}

\vspace{1cm}
Representaci'on del recurso de un paquete:
\vspace{1cm}

\begin{lstlisting}[style=json]
{
  "packages": [
      {"name": "package_name1"},
      {"name": "package_name2"}
  ],
  "hosts": [ 
      {"name": "Host_", "range": "[1-3]"}, 
      {"name": "database_001"}
  ],
  "tags": [
    {"name": "db"},
    {"name": "web"}
  ]
}
\end{lstlisting}

Ejemplo:

\medskip
\begin{lstlisting}[style=json]
{
  "packages": [
      {"name": "git"},
      {"name": "nginx"}
  ],
  "hosts": [ 
      {"name": "Host_", "range": "[1-3]"}
  ]
}
\end{lstlisting}

\subsubsection{Ejemplos de peticiones HTTP}

\paragraph{Instalar uno o mas paquetes - POST /api/virtshell/v1/install\_packages} ~\\

\begin{lstlisting}[style=json]
curl -sv -X PUT \
  -H 'accept: application/json' \
  -H "Content-Type: text/plain" \
  -H 'X-VirtShell-Authorization: UserId:Signature' \
  -d '{ "packages": [{"name": "git"}, {"name": "nginx"}],
        "hosts": [{"name": "WebServer_", "range": "[1-3]"}]}' \
   'http://localhost:8080/api/virtshell/v1/install_packages'
\end{lstlisting}

\vspace{1cm}
Respuesta:
\vspace{1cm}

\begin{lstlisting}[style=json]
HTTP/1.1 202 Accepted
Content-Type: application/json
{ "install_package": "accepted" }
\end{lstlisting}

\paragraph{Actualizar uno o mas paquetes - POST /api/virtshell/v1/upgrade\_packages} ~\\

\begin{lstlisting}[style=json]
curl -sv -X PUT \
  -H 'accept: application/json' \
  -H "Content-Type: text/plain" \
  -H 'X-VirtShell-Authorization: UserId:Signature' \
  -d '{ "packages": [{"name": "git"}, {"name": "nginx"}, {"name": "mc"}],
        "hosts": [{"name": "WebServer_", "range": "[1-3]"}]}' \
   'http://localhost:8080/api/virtshell/v1/upgrade_packages'
\end{lstlisting}

\vspace{1cm}
Respuesta:
\vspace{1cm}

\begin{lstlisting}[style=json]
HTTP/1.1 202 Accepted
Content-Type: application/json
{ "install_package": "accepted" }
\end{lstlisting}

\paragraph{Remover uno o mas paquetes - POST /api/virtshell/v1/remove\_packages} ~\\

\begin{lstlisting}[style=json]
curl -sv -X PUT \
  -H 'accept: application/json' \
  -H "Content-Type: text/plain" \
  -H 'X-VirtShell-Authorization: UserId:Signature' \
  -d '{ "packages": [{"name": "apache2"}],
        "hosts": [{"name": "WebServer_", "range": "[1-3]"}]}' \
   'http://localhost:8080/api/virtshell/v1/remove_packages'
\end{lstlisting}

\vspace{1cm}
Respuesta:
\vspace{1cm}

\begin{lstlisting}[style=json]
HTTP/1.1 202 Accepted
Content-Type: application/json
{ "install_package": "accepted" }
\end{lstlisting}
\subsection{Properties}
Representan propiedades de configuraci'on de las m'aquinas virtuales o contenedores. Los metodos soportados son:

\begin{center}
 \captionof{table}{Métodos HTTP para properties}
 \begin{tabular}{| l | l | l | l |}
 \hline
  \rowcolor{blueapi}
  \textbf{Acci'on} & \textbf{Metodo HTTP} & \textbf{Solicitud HTTP} & \textbf{Descripci'on} \\ [0.5ex] 
  \hline\hline
  get & GET & /properties/ & Install one or more packages. \\ [1ex] 
  \hline
\end{tabular}
\end{center}

\vspace{1cm}
Representaci'on del recurso de un paquete:
\vspace{1cm}

\begin{lstlisting}[style=json]
{
  "properties": [
      {"name": "propertie_name1"},
      {"name": "propertie_name2"}
  ],
  "hosts": [ 
      {"name": "Host_", "range": "[1-3]"}, 
      {"name": "database_001"}
  ],
  "tags": [
    {"name": "db"},
    {"name": "web"}
  ]
}
\end{lstlisting}

Ejemplo:

\medskip
\begin{lstlisting}[style=json]
{
  "properties": [
      {"name": "memory"},
      {"name": "cpu"}
  ],
  "hosts": [ 
      {"name": "Host_", "range": "[1-3]"}
  ]
}
\end{lstlisting}

\subsubsection{Ejemplos de peticiones HTTP}

\paragraph{Obtener una o mas propiedades de una unica instancia - POST /api/virtshell/v1/properties} ~\\

\begin{lstlisting}[style=json]
curl -sv -X GET \
  -H 'accept: application/json' \
  -H "Content-Type: text/plain" \
  -H 'X-VirtShell-Authorization: UserId:Signature' \
  -d '{ "properties": [{"name": "memory"}, {"name": "cpu"}],
        "hosts": [{"name": "WebServer"}]}' \
   'http://localhost:8080/api/virtshell/v1/properties'
\end{lstlisting}

\vspace{1cm}
Respuesta:
\vspace{1cm}

\begin{lstlisting}[style=json]
HTTP/1.1 202 OK
Content-Type: application/json
{
  "id": "kj5436c0-dc94-13tg-82ce-9992b5d5c51b",
  "name": "Database001",
  "memory": 1024
}
\end{lstlisting}

\paragraph{Obtener una o mas propiedades de una o mas instancias por tag - POST /api/virtshell/v1/properties} ~\\

\begin{lstlisting}[style=json]
curl -sv -X GET \
  -H 'accept: application/json' \
  -H "Content-Type: text/plain" \
  -H 'X-VirtShell-Authorization: UserId:Signature' \
  -d '{ "properties": [{"name": "memory"}, {"name": "cpu"}],
        "tag": [{"name": "web"}]}' \
   'http://localhost:8080/api/virtshell/v1/properties'
\end{lstlisting}

\vspace{1cm}
Respuesta:
\vspace{1cm}

\begin{lstlisting}[style=json]
HTTP/1.1 202 OK
Content-Type: application/json
{
  properties: [
    {
     "id": "kj5436c0-dc94-13tg-82ce-9992b5d5c51b",
     "name": "WebServerPhp001",
     "memory": 1024,
     "cpu": 2
    },
    {
     "id": "591b3828-7aaf-4833-a94c-ad0df44d59a4",
     "name": "WebServerPhp002",
     "memory": 1024,
     "cpu": 1  
    }
  ]
}
\end{lstlisting}

\paragraph{Obtener una o mas propiedades de una o mas instancias usando como prefijo un rango - POST /api/virtshell/v1/properties} ~\\

\begin{lstlisting}[style=json]
curl -sv -X GET \
  -H 'accept: application/json' \
  -H "Content-Type: text/plain" \
  -H 'X-VirtShell-Authorization: UserId:Signature' \
  -d '{ "properties": [{"name": "memory"}, {"name": "cpu"}],
        {"name": "Database00", "range": "[1-3]"}]}' \
   'http://localhost:8080/api/virtshell/v1/properties'
\end{lstlisting}

\vspace{1cm}
Respuesta:
\vspace{1cm}

\begin{lstlisting}[style=json]
HTTP/1.1 202 OK
Content-Type: application/json
{
  properties: [
    {
     "id": "kj5436c0-dc94-13tg-82ce-9992b5d5c51b",
     "name": "Database001",
     "memory": 4024,
     "cpu": 2
    },
    {
     "id": "591b3828-7aaf-4833-a94c-ad0df44d59a4",
     "name": "Database002",
     "memory": 4024,
     "cpu": 1  
    },
    {
     "id": "f7c81039-5c88-423b-8b0d-c124483d586b",
     "name": "Database003",
     "memory": 4024,
     "cpu": 3  
    }
  ]  
}
\end{lstlisting}
\subsection{Groups}
Representan los grupos registrados en VirtShell. Los metodos soportados son:

\begin{center}
 \begin{tabular}{| l | l | l | l |}
 \hline
  \rowcolor{blueapi}
  \textbf{Acci'on} & \textbf{Metodo HTTP} & \textbf{Solicitud HTTP} & \textbf{Descripci'on} \\ [0.5ex] 
  \hline\hline
  get & GET & /users/id & Gets one group by ID. \\
  \hline
  list & GET & /hosts & Retrieves the list of groups. \\  
  \hline
  create & POST & /users/ & creates a new group. \\
  \hline
  delete & DELETE & /users/id & Deletes an existing group. \\
  \hline
\end{tabular}
\end{center}

\vspace{1cm}
Representaci'on del recurso de un grupo:
\vspace{1cm}

\begin{lstlisting}[style=json]
{
  "uuid": "ab8076c0-db91-11e2-82ce-0002a5d5c51b",
  "name": "web_development_team",
  "users": [ ... list of members of the group ...],  
  "created":[ {"at":"timestamp"}, {"by":user_id}]
}
\end{lstlisting}

Ejemplo:

\medskip
\begin{lstlisting}[style=json]
{
  "uuid": "ab8076c0-db91-11e2-82ce-0002a5d5c51b",
  "name": "web_development_team",
  "users": [ 
      {"username": "user1", "id": "a146cae4-8c90-11e5-8994-feff819cdc9f"},
      {"username": "user2", "id": "a146d00c-8c90-11e5-8994-feff819cdc9f"}
  ]
  "created":[{"at":"1447696674"}, {"by":"a379e8e6-8c8b-11e5-8994-feff819cdc9f"}]
}
\end{lstlisting}

\subsubsection{Ejemplos de peticiones HTTP}

\paragraph{Crear un nuevo grupo - POST /api/virtshell/v1/grupos} ~\\

\begin{lstlisting}[style=json]
curl -X POST \
  -H 'accept: application/json' \
  -H 'X-VirtShell-Authorization: UserId:Signature' \
  -H "Content-Type: multipart/form-data" \
  -d '{"name": "database_team"}' \
  'http://<host>:<port>/api/virtshell/v1/groups'
\end{lstlisting}

\vspace{1cm}
Respuesta:
\vspace{1cm}

\begin{lstlisting}[style=json]
HTTP/1.1 200 OK
Content-Type: application/json
{ "create": "success" }
\end{lstlisting}

\paragraph{Obtener un grupo - GET /api/virtshell/v1/groups/:id} ~\\

\begin{lstlisting}[style=json]
curl -sv -H 'accept: application/json' 
     -H 'X-VirtShell-Authorization: UserId:Signature' \ 
     'http://<host>:<port>/api/virtshell/v1/groups/?id=ab8076c0-db91-11e2-82ce-0002a5d5c51b'
\end{lstlisting}

\vspace{1cm}
Respuesta:
\vspace{1cm}

\begin{lstlisting}[style=json]
HTTP/1.1 200 OK
Content-Type: application/json
{
  "uuid": "ab8076c0-db91-11e2-82ce-0002a5d5c51b",
  "name": "web_development_team",
  "users": [ 
      {"username": "user1", "id": "a146cae4-8c90-11e5-8994-feff819cdc9f"},
      {"username": "user2", "id": "a146d00c-8c90-11e5-8994-feff819cdc9f"}
  ]
  "created":[{"at":"1447696674"}, {"by":"a379e8e6-8c8b-11e5-8994-feff819cdc9f"}]
}
\end{lstlisting}

\paragraph{Obtener todos los grupos - GET /api/virtshell/v1/groups} ~\\

\begin{lstlisting}[style=json]
curl -sv -H 'accept: application/json' 
     -H 'X-VirtShell-Authorization: UserId:Signature' \ 
     'http://localhost:8080/api/virtshell/v1/groups'
\end{lstlisting}

\vspace{1cm}
Respuesta:
\vspace{1cm}

\begin{lstlisting}[style=json]
HTTP/1.1 200 OK
Content-Type: application/json
{
  "groups": [
    {
      "uuid": "ab8076c0-db91-11e2-82ce-0002a5d5c51b",
      "name": "web_development_team",
      "users": [ 
          {"username": "user1", "id": "a146cae4-8c90-11e5-8994-feff819cdc9f"},
          {"username": "user2", "id": "a146d00c-8c90-11e5-8994-feff819cdc9f"}
      ],     
      "created":[{"at":"1447696833"}, {"by":"d2372efa-8c8b-11e5-8994-feff819cdc9f"}]
    },
    {
      "uuid": "a379f19c-8c8b-11e5-8994-feff819cdc9f",
      "name": "math_team",
      "users": [ 
          {"username": "user3", "id": "a146cae4-8c90-11e5-8994-feff819cdc9f"}
      ],     
      "created":[{"at":"1421431233"}, {"by":"18489280-8c91-11e5-8994-feff819cdc9f"}]
    },
    {
      "uuid": "a379f3d6-8c8b-11e5-8994-feff819cdc9f",
      "name": "chemical_team",
      "users": [ 
          {"username": "user4", "id": "F8489280-8c91-11e5-8994-feff819cdc9f"},
          {"username": "user5", "id": "18489780-8c91-11e5-8994-feff819cdc9f"}
      ],       
      "created":[{"at":"1424109633"}, {"by":"d2373576-8c8b-11e5-8994-feff819cdc9f"}]
    },        
}  
\end{lstlisting}

\paragraph{Eliminar un grupo - DELETE /api/virtshell/v1/groups/:id} ~\\

Para eliminar un grupo se debe tener en cuenta que no debe tener usuarios asociados a el.

\begin{lstlisting}[style=json]
curl -sv -X DELETE \
   -H 'accept: application/json' \
   -H 'X-VirtShell-Authorization: UserId:Signature' \
   'http://localhost:8080/api/virtshell/v1/groups?id=73cff0b0-8c8e-11e5-8994-feff819cdc9f'
\end{lstlisting}

\vspace{1cm}
Respuesta:
\vspace{1cm}

\begin{lstlisting}[style=json]
HTTP/1.1 200 OK
Content-Type: application/json
```
```json
{ "delete": "success" }
\end{lstlisting}

\subsection{Users}
Representan los usuarios registrados en VirtShell. Los metodos soportados son:

\begin{center}
 \captionof{table}{Métodos HTTP para users}
 \begin{tabular}{| l | l | l | l |}
 \hline
  \rowcolor{blueapi}
  \textbf{Acci'on} & \textbf{Metodo HTTP} & \textbf{Solicitud HTTP} & \textbf{Descripci'on} \\ [0.5ex] 
  \hline\hline
  get & GET & /users/:name & Gets one user by ID. \\
  \hline
  create & POST & /users/ & creates a new user. \\
  \hline
  list & GET & /users & Retrieves the list of users. \\  
  \hline
  delete & DELETE & /users/:name & Deletes an existing user. \\
  \hline  
  update & PUT & /users/:name & Updates an existing user. \\ [1ex]  
  \hline
\end{tabular}
\end{center}

\vspace{1cm}
Representaci'on del recurso de un usuario:
\vspace{1cm}

\begin{lstlisting}[style=json]
{
  "uuid": "ab8076c0-db91-11e2-82ce-0002a5d5c51b",
  "username": "virtshell",
  "type": "system/regular",
  "login": "user@mail.com",
  "groups": [ ... list of users ...],
  "created": {"at": timestamp, "by": user_uuid},
  "modified": {"at": timestamp, "by": user_uuid}
}
\end{lstlisting}

Ejemplo:

\medskip
\begin{lstlisting}[style=json]
{
  "uuid": "ab8076c0-db91-11e2-82ce-0002a5d5c51b",
  "username": "virtshell",
  "type": "system/regular",
  "login": "user@mail.com",
  "groups": [ {"name": "web_development_team"},
              {"name": "production"}
  ],
  "created": {"at":"1429207233", "by":"92d30f0c-8c9c-11e5-8994-feff819cdc9f"},
  "modified": {"at":"1529207233", "by":"92d31132-8c9c-11e5-8994-feff819cdc9f"}
}
\end{lstlisting}

\subsubsection{Ejemplos de peticiones HTTP}

\paragraph{Crear un nuevo usuario - POST /api/virtshell/v1/users} ~\\

\begin{lstlisting}[style=json]
curl -X POST \
  -H 'accept: application/json' \
  -H 'X-VirtShell-Authorization: UserId:Signature' \
  -H "Content-Type: multipart/form-data" \
  -d {
       "username": "virtshell", 
       "type": "system/regular",
       "login": "user@mail.com",
       "groups": [ {"name": "web_development_team"},
                   {"name": "production"} ]
      } \
  'http://<host>:<port>/api/virtshell/v1/users'
\end{lstlisting}

\vspace{1cm}
Respuesta:
\vspace{1cm}

\begin{lstlisting}[style=json]
HTTP/1.1 200 OK
Content-Type: application/json
{ "create": "success" }
\end{lstlisting}

\paragraph{Obtener un usuario - GET /api/virtshell/v1/users/:name} ~\\

\begin{lstlisting}[style=json]
curl -sv -H 'accept: application/json' 
     -H 'X-VirtShell-Authorization: UserId:Signature' \ 
     'http://<host>:<port>/api/virtshell/v1/users/virtshell'
\end{lstlisting}

\vspace{1cm}
Respuesta:
\vspace{1cm}

\begin{lstlisting}[style=json]
HTTP/1.1 200 OK
Content-Type: application/json
{
  "uuid": "ab8076c0-db91-11e2-82ce-0002a5d5c51b",
  "username": "virtshell",
  "type": "system/regular",
  "login": "user@mail.com",
  "groups": [ {"name": "web_development_team"},
              {"name": "production"}
  ],
  "created": {"at":"1429207233", "by":"92d30f0c-8c9c-11e5-8994-feff819cdc9f"},
  "modified": {"at":"1529207233", "by":"92d31132-8c9c-11e5-8994-feff819cdc9f"}
}
\end{lstlisting}

\paragraph{Actualizar un usuario - PUT /api/virtshell/v1/users/:name} ~\\

\begin{lstlisting}[style=json]
curl -sv -X PUT \
  -H 'accept: application/json' \
  -H 'X-VirtShell-Authorization: UserId:Signature' \
  -H "Content-Type: multipart/form-data" \
  -d '{"type": "system",
       "groups": [{"uuid": "a146cae4-8c90-11e5-8994-feff819cdc9f"},
                  {"uuid": "a146d00c-8c90-11e5-8994-feff819cdc9f"}]}' \
   'http://localhost:8080/api/virtshell/v1/file/virtshell'
\end{lstlisting}

\vspace{1cm}
Respuesta:
\vspace{1cm}

\begin{lstlisting}[style=json]
HTTP/1.1 200 OK
Content-Type: application/json

{ "update": "success" }
\end{lstlisting}


\paragraph{Eliminar un usuario - DELETE /api/virtshell/v1/users/:name} ~\\

\begin{lstlisting}[style=json]
curl -sv -X DELETE \
   -H 'accept: application/json' \
   -H 'X-VirtShell-Authorization: UserId:Signature' \
   'http://localhost:8080/api/virtshell/v1/fles/virtshell'
\end{lstlisting}

\vspace{1cm}
Respuesta:
\vspace{1cm}

\begin{lstlisting}[style=json]
HTTP/1.1 200 OK
Content-Type: application/json
```
```json
{ "delete": "success" }
\end{lstlisting}


% VirtShell is a multi-user framework that is based on the Unix permissions concepts to provide security.

% VirtShell provides mechanisms to control access by  limiting the types of
% resource access that can be made. Access is permitted or denied depending on
% several factors, one of which is the type of access requested. Several different
% types of operations may be controlled:

% Read. Read from the resouce.
% Write. Write or rewrite of resoures.
% Execute. Load the resource into host and execute it.

% Here is a quick breakdown of the access that the three basic permission types grant a user.

% Read
% ----
% Read permission allows a user to view the contents of any resource in VirtShell.

% Write
% -----
% Write permission allows a user to create, modify and delete whatever resources.

% Execute
% -------
% Execute permission allows a user to execute virtual machines or containers, for example: start, stop, pause, snapshot. (the user must also have read permission). 
\subsection{Provisioners}
Representan los scripts que aprovisionan las m'aquinas virtuales o los contenedores. Los metodos soportados son:

\begin{center}
 \begin{tabular}{| l | l | l | l |}
 \hline
  \rowcolor{blueapi}
  \textbf{Acci'on} & \textbf{Metodo HTTP} & \textbf{Solicitud HTTP} & \textbf{Descripci'on} \\ [0.5ex] 
  \hline\hline
  get & GET & /provisioners/id & Gets one provisioner by ID. \\
  \hline
  list & GET & /provisioners & Retrieves the list of provisioners. \\
  \hline  
  create & POST & /provisioners/ & Creates a new provisioner. \\
  \hline
  delete & DELETE & /provisioners/id & Deletes an existing host. \\
  \hline  
  update & PUT & /provisioners/id & Updates an existing provisioner. \\ [1ex] 
  \hline
\end{tabular}
\end{center}

Representaci'on del recurso de un provisioner:

\medskip
\begin{lstlisting}[style=json]
{
  "uuid": string,
  "name": string,
  "description": string,
  "version": string,
  "builder": url,
  "how_to_run": string,
  "tag": string,
  "depends": [ ... list of dependencies necessary for the builder ... ],
  "files": [ ... list of files necessary for the builder ... ],
  "templates": [ ... list of templates necessary for the builder ...],
  "permissions" : string,
  "created": {"at": timestamp, "by": string},
  "modified": {"at": timestamp, "by": string}
}

\end{lstlisting}

Ejemplo:

\medskip
\begin{lstlisting}[style=json]
{
  "uuid": "420aa3f0-8d96-11e5-8994-feff819cdc9f",
  "name": "backend-services-provisioner",
  "builder": "https://<host>:<port>/api/virtshell/v1/files/builders/director-backend.sh",
  "how_to_run": "sh",
  "tag": "backend",
  "depends": [
      {"provisioner_name": "db-users", "version": "2.0.0"},
      {"provisioner_name": "db-transactional"}
  ],
  "files": [
      {"path": "https://<host>:<port>/api/virtshell/v1/files/queues/queue_mail_transform.py}
  ],
  "templates": [
      {"path": "https://<host>:<port>/api/virtshell/v1/files/queues/queue_mail_config.xml}
  ],
  "permissions" : "rwxrw----",       
  "created": {"at":"1429207233", "by":"92d30f0c-8c9c-11e5-8994-feff819cdc9f"},
  "modified": {"at":"1529207233", "by":"92d31132-8c9c-11e5-8994-feff819cdc9f"}
}
\end{lstlisting}

\subsubsection{Ejemplos de peticiones HTTP}

\paragraph{Crear un nuevo provisioner - POST /virtshell/api/v1/provisioners} ~\\


\begin{lstlisting}[style=json]
curl -sv -X POST \
  -H 'accept: application/json' \
  -H 'X-VirtShell-Authorization: UserId:Signature' \
  -d '{"name": "backend-services-provisioner",
       "builder": "https://<host>:<port>/api/virtshell/v1/files/builders/director-backend.sh",
       "how_to_run": "sh",
       "tag": "backend",
       "depends": [
            {"provisioner_name": "db-users", "version": "2.0.0"},
            {"provisioner_name": "db-transactional"}
        ],
       "files": [
            {"path": "https://<host>:<port>/api/virtshell/v1/files/queues/queue_mail_transform.py}
        ],
      "templates": [
            {"path": "https://<host>:<port>/api/virtshell/v1/files/queues/queue_mail_config.xml}
        ],
      "permissions" : "rwxrw----",       
       "created": {"at":"1429207233", "by":"92d30f0c-8c9c-11e5-8994-feff819cdc9f"},
       "modified": {"at":"1529207233", "by":"92d31132-8c9c-11e5-8994-feff819cdc9f"}
      }' \
   'http://localhost:8080/virtshell/api/v1/provisioners'
\end{lstlisting}

Response:

\begin{lstlisting}[style=json]
HTTP/1.1 200 OK
Content-Type: application/json
{ "create": "success" }
\end{lstlisting}

\paragraph{Obtener un provisioner- GET /virtshell/api/v1/provisioners/:id} ~\\

\begin{lstlisting}[style=json]
curl -sv -H 'accept: application/json' 
     -H 'X-VirtShell-Authorization: UserId:Signature' \ 
     'http://localhost:8080/api/virtshell/v1/provisioners?id=420aa3f0-8d96-11e5-8994-feff819cdc9f'
\end{lstlisting}

Response:

\begin{lstlisting}[style=json]
HTTP/1.1 200 OK
Content-Type: application/json
{
  "uuid": "420aa3f0-8d96-11e5-8994-feff819cdc9f",
  "name": "backend-services-provisioner",
  "builder": "https://<host>:<port>/api/virtshell/v1/files/builders/director-backend.sh",
  "how_to_run": "sh",
  "tag": "backend",
  "depends": [
      {"provisioner_name": "db-users", "version": "2.0.0"},
      {"provisioner_name": "db-transactional"}
  ],
  "files": [
      {"path": "https://<host>:<port>/api/virtshell/v1/files/queues/queue_mail_transform.py}
  ],
  "templates": [
      {"path": "https://<host>:<port>/api/virtshell/v1/files/queues/queue_mail_config.xml}
  ],
  "permissions" : "rwxrwx---",        
  "created": {"at":"1429207233", "by":"92d30f0c-8c9c-11e5-8994-feff819cdc9f"},
  "modified": {"at":"1529207233", "by":"92d31132-8c9c-11e5-8994-feff819cdc9f"}
}
\end{lstlisting}

\paragraph{Obtener todos los provisioners - GET /virtshell/api/v1/provisioners} ~\\

\begin{lstlisting}[style=json]
curl -sv -H 'accept: application/json' 
     -H 'X-VirtShell-Authorization: UserId:Signature' \ 
     'http://localhost:8080/api/virtshell/v1/provisioners'
\end{lstlisting}

Response:

\begin{lstlisting}[style=json]
HTTP/1.1 200 OK
Content-Type: application/json
{
  "provisioners": [
    { 
      "uuid": "420a9fae-8d96-11e5-8994-feff819cdc9f",
      "name": "backend-services-provisioner",
      "builder": "https://<host>:<port>/api/virtshell/v1/files/builders/director-backend.sh",
      "how_to_run": "sh",
      "tag": "backend",
      "depends": [
          {"provisioner_name": "db-users", "version": "2.0.0"},
          {"provisioner_name": "db-transactional"}
      ],
      "files": [
          {"path": "https://<host>:<port>/api/virtshell/v1/files/queues/queue_mail_transform.py}
      ],
      "templates": [
          {"path": "https://<host>:<port>/api/virtshell/v1/files/queues/queue_mail_config.xml}
      ],
      "permissions" : "rwxr----",      
      "created": {"at":"1429207233", "by":"92d30f0c-8c9c-11e5-8994-feff819cdc9f"},
      "modified": {"at":"1529207233", "by":"92d31132-8c9c-11e5-8994-feff819cdc9f"}
    },
    { 
      "uuid": "420a9fae-8d96-11e5-8994-feff819cdc9f",
      "name": "db-transactional",
      "builder": "https://<host>:<port>/api/virtshell/v1/files/databases/director-dbt.sh",
      "how_to_run": "sh",
      "tag": "dbt",
      "permissions" : "rwxrwxrw-",        
      "created": {"at":"1429207233", "by":"420aa2c4-8d96-11e5-8994-feff819cdc9f"},
      "modified": {"at":"1529207233", "by":"92d31132-8c9c-11e5-8994-feff819cdc9f"}
    }    
  ]
} 
\end{lstlisting}

\paragraph{Actualizar un provisioner - PUT /virtshell/api/v1/provisioners/:id} ~\\

\begin{lstlisting}[style=json]
curl -sv -X PUT \
  -H 'accept: application/json' \
  -H 'X-VirtShell-Authorization: UserId:Signature' \
  -d '{"files": [{"path": "https://<host>:<port>/api/virtshell/v1/files/mysql/my.cnf}],
       "permissions" : "rwxrw----"}' \
   'http://localhost:8080/api/virtshell/v1/provisioners?id=ab8076c0-db91-11e2-82ce-0002a5d5c51b'
\end{lstlisting}

Response:

\begin{lstlisting}[style=json]
HTTP/1.1 200 OK
Content-Type: application/json

{ "update": "success" }
\end{lstlisting}

\paragraph{Eliminar un provisioner - DELETE /virtshell/api/v1/provisioners/:id} ~\\

\begin{lstlisting}[style=json]
curl -sv -X DELETE \
   -H 'accept: application/json' \
   -H 'X-VirtShell-Authorization: UserId:Signature' \
   'http://localhost:8080/api/virtshell/v1/provisioners?id=ab8076c0-db91-11e2-82ce-0002a5d5c51b'
\end{lstlisting}

Response:

\begin{lstlisting}[style=json]
HTTP/1.1 200 OK
Content-Type: application/json
```
```json
{ "delete": "success" }
\end{lstlisting}
\subsection{Instances}
Representan las instancias de las m'aquinas virtuales o los contenedores. Los metodos soportados son:

\begin{center}
 \captionof{table}{Métodos HTTP para instances}
 \begin{tabular}{| l | l | l | l |}
 \hline
  \rowcolor{blueapi}
  \textbf{Acci'on} & \textbf{Metodo HTTP} & \textbf{Solicitud HTTP} & \textbf{Descripci'on} \\ [0.5ex] 
  \hline\hline
  get & GET & /provisioners/:name & Gets one provisioner by ID. \\
  \hline
  list & GET & /provisioners & Retrieves the list of provisioners. \\
  \hline  
  create & POST & /provisioners/ & Creates a new provisioner. \\
  \hline
  delete & DELETE & /provisioners/:name & Deletes an existing host. \\ [1ex] 
  \hline
\end{tabular}
\end{center}

Representaci'on del recurso de un provisioner:

\medskip
\begin{lstlisting}[style=json]
{
  "uuid": string,
  "name": string,
  "memory": numeric,
  "cpus": numeric,
  "hdsize": string,
  "description": string, 
  "enviroment": string,
  "image": string,
  "provisioner": string,
  "host_type": string,
  "ipv4": string,
  "ipv6": string,
  "driver": string,
  "permissions": string,
  "created": {"at": timestamp, "by": string},
  "modified": {"at": timestamp, "by": string}
}
\end{lstlisting}

Ejemplo:

\medskip
\begin{lstlisting}[style=json]
{
  "uuid": "ab8076c0-db91-11e2-82ce-0002a5d5c51b",
  "name": "transactional_log",
  "memory": 1024,
  "cpus": 2,
  "hdsize": "2GB",
  "description": "Server transactional only for store logs", 
  "enviroment": "Enviroment name to which it belongs",
  "image": "ubuntu_server_14.04.2_amd64",
  "provisioner": "all_backend",
  "host_type": "GeneralPurpose | ComputeOptimized | MemoryOptimized | StorageOptimized",
  "ipv4": "172.16.56.104",
  "ipv6": "FE80:0000:0000:0000:0202:B3FF:FE1E:8329",
  "driver": "lxc | virtualbox | vmware | ec2 | kvm | docker",
  "permissions": "xwrxwrxwr",
  "created": {"at":"1429207233", "by":"92d30f0c-8c9c-11e5-8994-feff819cdc9f"},
  "modified": {"at":"1529207233", "by":"92d31132-8c9c-11e5-8994-feff819cdc9f"}
}
\end{lstlisting}

\subsubsection{Ejemplos de peticiones HTTP}

\paragraph{Crear una nueva instance - POST /api/virtshell/v1/instances} ~\\


\begin{lstlisting}[style=json]
curl -sv -X POST \
  -H 'accept: application/json' \
  -H 'X-VirtShell-Authorization: UserId:Signature' \
  -d '{ "name": "transactional_log",
        "memory": 1024,
        "cpus": 2,
        "hdsize": "2GB",
        "enviroment": "development_co",
        "image": "ubuntu_server_14.04.2_amd64",
        "description": "Server transactional only for store logs", 
        "provisioner": "all_backend",
        "host_type": "GeneralPurpose",
        "driver": "lxc"
      }' \
   'http://localhost:8080/virtshell/api/v1/instances'
\end{lstlisting}

Response:

\begin{lstlisting}[style=json]
HTTP/1.1 200 OK
Content-Type: application/json
{ "create": "in progress" }
\end{lstlisting}

\paragraph{Obtener un instance- GET /api/virtshell/v1/instances/:name} ~\\

\begin{lstlisting}[style=json]
curl -sv -H 'accept: application/json' 
     -H 'X-VirtShell-Authorization: UserId:Signature' \ 
     'http://<host>:<port>/api/virtshell/v1/instances/orders_colombia'
\end{lstlisting}

Response:

\begin{lstlisting}[style=json]
HTTP/1.1 200 OK
Content-Type: application/json
{
  "uuid": "ab8076c0-db91-11e2-82ce-0002a5d5c51b",
  "name": "transactional_log",
  "memory": 1024,
  "cpus": 2,
  "hdsize": "2GB",
  "enviroment": "development_co",
  "image": "ubuntu_server_14.04.2_amd64",
  "description": "Server transactional only for store logs", 
  "provisioner": "all_backend",
  "host_type": "GeneralPurpose",
  "drive": "lxc",
  "created": {"at":"1429207233", "by":"92d30f0c-8c9c-11e5-8994-feff819cdc9f"},
  "modified": {"at":"1529207233", "by":"cf744732-8f12-11e5-8994-feff819cdc9f"}
  }
\end{lstlisting}

\paragraph{Obtener todos las instances - GET /api/virtshell/v1/instances} ~\\

\begin{lstlisting}[style=json]
curl -sv -H 'accept: application/json' 
     -H 'X-VirtShell-Authorization: UserId:Signature' \ 
     'http://localhost:8080/api/virtshell/v1/instances'
\end{lstlisting}

Response:

\begin{lstlisting}[style=json]
HTTP/1.1 200 OK
Content-Type: application/json
{
  "instances": [
    {
      "uuid": "ab8076c0-db91-11e2-82ce-0002a5d5c51b",
      "name": "transactional_log",
      "memory": 1024,
      "cpus": 2,
      "hdsize": "2GB",
      "enviroment": "development_co",
      "image": "ubuntu_server_14.04.2_amd64",
      "description": "Server transactional only for store logs", 
      "provisioner": "all_backend",
      "host_type": "GeneralPurpose",
      "drive": "lxc",
      "permissions": "xwrxwrxwr",
      "created": {"at":"1429207233", "by":"92d30f0c-8c9c-11e5-8994-feff819cdc9f"},
      "modified": {"at":"1529207233", "by":"cf744732-8f12-11e5-8994-feff819cdc9f"}
    },
    { 
      "uuid": "cf744476-8f12-11e5-8994-feff819cdc9f",
      "name": "orders_colombia",
      "memory": 2024,
      "cpus": 2,
      "hdsize": "4GB",
      "image": "ubuntu_server_14.04.2_amd64",
      "description": "Server transactional dedicated to receive orders", 
      "enviroment": "development_mx",
      "provisioner": "all_backend",
      "host_type": "StorageOptimized",
      "drive": "docker",
      "permissions": "xwrxwrxwr",
      "created": {"at":"1429207233", "by":"92d30f0c-8c9c-11e5-8994-feff819cdc9f"},
      "modified": {"at":"1529207233", "by":"92d31132-8c9c-11e5-8994-feff819cdc9f"}
    }    
  ]
} 
\end{lstlisting}

\paragraph{Eliminar una instance - DELETE /api/virtshell/v1/instances/:nae} ~\\

\begin{lstlisting}[style=json]
curl -sv -X DELETE \
   -H 'accept: application/json' \
   -H 'X-VirtShell-Authorization: UserId:Signature' \
   'http://<host>:<port>/api/virtshell/v1/instances/orders_colombia'
\end{lstlisting}

Response:

\begin{lstlisting}[style=json]
HTTP/1.1 200 OK
Content-Type: application/json
```
```json
{ "delete": "in progress" }
\end{lstlisting}

\section{API Calls}

\subsection{Start Instance}

Permite iniciar una instancia.

\paragraph{Iniciar una instance - \\ POST /virtshell/api/v1/instances/start\_instance/:id} ~\\


\begin{lstlisting}[style=json]
curl -sv -X POST \
  -H 'accept: application/json' \
  -H 'X-VirtShell-Authorization: UserId:Signature' \
   'http://localhost:8080/virtshell/api/v1/instances/start\_instance/420aa3f0-8d96-11e5-8994-feff819cdc9f'
\end{lstlisting}

Response:

\begin{lstlisting}[style=json]
HTTP/1.1 200 OK
Content-Type: application/json
{ "start": "success" }
\end{lstlisting}


\subsection{Stop Instance}

Permite detener una instancia.

\paragraph{Detener una instancia - \\ POST /virtshell/api/v1/instances/stop\_instance/:id} ~\\

\begin{lstlisting}[style=json]
curl -sv -X POST \
  -H 'accept: application/json' \
  -H 'X-VirtShell-Authorization: UserId:Signature' \
   'http://localhost:8080/virtshell/api/v1/instances/stop\_instance/420aa3f0-8d96-11e5-8994-feff819cdc9f'
\end{lstlisting}

Response:

\begin{lstlisting}[style=json]
HTTP/1.1 200 OK
Content-Type: application/json
{ "stop": "success" }
\end{lstlisting}


\subsection{Restart Instance}

Permite reiniciar una instancia.

\paragraph{Reiniciar una instancia - \\ POST /virtshell/api/v1/instances/restart\_instance/:id} ~\\

\begin{lstlisting}[style=json]
curl -sv -X POST \
  -H 'accept: application/json' \
  -H 'X-VirtShell-Authorization: UserId:Signature' \
   'http://localhost:8080/virtshell/api/v1/instances/restart\_instance/420aa3f0-8d96-11e5-8994-feff819cdc9f'
\end{lstlisting}

Response:

\begin{lstlisting}[style=json]
HTTP/1.1 200 OK
Content-Type: application/json
{ "restart": "success" }
\end{lstlisting}


\subsection{Clone Instance}

Permite clonar una instancia.

\paragraph{Clonar una instancia - \\ POST /virtshell/api/v1/instances/clone\_instance/:id} ~\\

\begin{lstlisting}[style=json]
curl -sv -X POST \
  -H 'accept: application/json' \
  -H 'X-VirtShell-Authorization: UserId:Signature' \
   'http://localhost:8080/virtshell/api/v1/instances/clone\_instance/420aa3f0-8d96-11e5-8994-feff819cdc9f'
\end{lstlisting}

Response:

\begin{lstlisting}[style=json]
HTTP/1.1 200 OK
Content-Type: application/json
{ "clone": "success" }
\end{lstlisting}

\subsection{Execute command}

Permite ejecutar un comando en una o mas instancias.

Representaci'on del recurso para ejecutar un comando:

\medskip
\begin{lstlisting}[style=json]
{
  "instances": [ ... list of instances names, patterns(*|[numeric:numeric]) or tags ...],
  "command": string,
  "created": {"at": timestamp, "by": string}
}
\end{lstlisting}

Ejemplo:

\medskip
\begin{lstlisting}[style=json]
{
  "instances": [
      {"name": "database\_server\_01"},
      {"name": "transactional\_server\_co"},      
      {"pattern": "web\_server*"},
      {"pattern": "grid\_[1:5]"},
      {"tag": "web"}
  ],
  "command": "apt-get upgrade",
  "created": {"at": timestamp, "by": string}
}
\end{lstlisting}

\paragraph{Ejecutar un comando en una o mas instancias - \\ POST /virtshell/api/v1/instances/execute\_command/} ~\\

\begin{lstlisting}[style=json]
curl -sv -X POST \
  -H 'accept: application/json' \
  -H 'X-VirtShell-Authorization: UserId:Signature' \
  -d '{ "instances": [
          {"name": "database\_server\_01"},
          {"name": "transactional\_server\_co"},          
          {"pattern": "web\_server*"},
          {"pattern": "grid\_server\_[1:5]"},
          {"tag": "web"}
        ],
        "command": "apt-get upgrade" }' \
  'http://localhost:8080/virtshell/api/v1/instances/execute\_command/'
\end{lstlisting}

Response:

\begin{lstlisting}[style=json]
HTTP/1.1 200 OK
Content-Type: application/json
{ "execute_command": "success" }
\end{lstlisting}


\subsection{Copy files}

Permite ejecutar copiar uno archivo en una o mas instancias.

Representaci'on del recurso para ejecutar un comando:

\medskip
\begin{lstlisting}[style=json]
{
  "path": string,
  "destination": string,
  "instances": [ ... list of instances names, patterns(*|[numeric:numeric]) or tags ...],
  "created": {"at": timestamp, "by": string}
}
\end{lstlisting}

Ejemplo:

\medskip
\begin{lstlisting}[style=json]
{
  "uuid_file": "0d832c60-7066-4d37-bd72-ce6ac4f61bcc",
  "destination": "$MYSQL_HOME/my.cnf"
  "instances": [
      {"name": "database\_server\_01"},
      {"name": "web\_server*"},
      {"name": "grid\_[1:5]"},
      {"name": "transactional\_server\_co"},
      {"tag": "web"}
  ]
}
\end{lstlisting}

\paragraph{Copiar un archivo en una o mas instancias - \\ POST /virtshell/api/v1/instances/copy\_files/} ~\\

\begin{lstlisting}[style=json]
curl -sv -X POST \
  -H 'accept: application/json' \
  -H 'X-VirtShell-Authorization: UserId:Signature' \
  -d '{ "uuid_file": "0d832c60-7066-4d37-bd72-ce6ac4f61bcc",
        "destination": "$MYSQL_HOME/my.cnf"
        "instances": [
            {"name": "database\_server\_01"},
            {"name": "web\_server*"},
            {"name": "grid\_[1:5]"},
            {"name": "transactional\_server\_co"},
            {"tag": "web"}
        ] }' \
  'http://localhost:8080/virtshell/api/v1/instances/copy\_files/'
\end{lstlisting}

Response:

\begin{lstlisting}[style=json]
HTTP/1.1 200 OK
Content-Type: application/json
{ "copy_files": "success" }
\end{lstlisting}


%\include{capitulo3}
\chapter{Recomendaciones}
\label{caprecomendaciones}

\begin{itemize}
\item Implementar una interfaz web que permita administrar los ambientes y maquinas virtuales.
\item Implementar los agentes de monitoreo de recursos.
\item Implementar algun mecanismo de seguridad que permita revisar las tramas que llegan y salen de las maquinas virtuales y los hosts.
\item Realizar un plan de pruebas funcionales para los ambientes que se aprovisionan.
\end{itemize}



\appendix
%% Cap'itulos incluidos despues del comando \appendix aparecen como ap'endices
%% de la tesis.
%\include{apendiceA}
%\include{apendiceB}
%\include{apendiceC}

%% Incluir la bibliograf'ia. Mirar el archivo "biblio.bib" para m'as detales
%% y un ejemplo.
\bibliography{biblio}

\end{document}
