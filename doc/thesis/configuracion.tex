% commpico.tex - Comandos para la tesis MAPiCO


\newcommand{\logeq}{\models \hspace{-0.16cm}|_{\Delta}�} 
\newcommand{\granlogeq}{\models \hspace{-0.1cm}|_{\Delta}�} 
\newcommand{\punto}[1]{\stackrel{.}{#1}} 
\newcommand{\redpi}{\longrightarrow} 
\newcommand{\redpid}{\Longrightarrow} 
\newcommand{\findem}{ \hfill \fbox{\rule{0mm}{1mm}}} 
\newcommand{\config}[2]{\left\langle #1;#2 \right\rangle} 
\newcommand{\reduce}[2]{#1 \redpi #2} 
\newcommand{\reduced}[2]{#1 \redpid #2} 
\newcommand{\judge}[2]{{\frac{#1}{#2}}} 
\newcommand{\semant}[1]{\left[ \! \left[{#1}\right] \! \right]} 

\newcommand{\configma}[6]{\left\langle #1;#2;#3;#4;#5;#6 \right\rangle} 

\newcommand{\confmaq}[8]{\left\langle #1,#2,#3,#4,#5,#6,#7,#8 \right\rangle} 
\newcommand{\reducenl}[2]{#1 \redpi \newline #2} 

\newcommand{\configmaq}[5]{\left\langle #1;#2;#3;#4;#5 \right\rangle} 




%mensaje prima general
\newcommand{\genmessprime}{�I' \lhd l_{i}�:[\widetilde{K}].\vec{P}}
%objeto prima general
\newcommand{\genobjprime} {(I',J') \rhd [l_{1}:(\widetilde{x_{1}�})\vec{P_1}, \ldots , l_{m}:(\widetilde{x_{n}�})\vec{P_m}]}

%mensaje general
\newcommand{\genmess}{�I \lhd l_{i}�:[\widetilde{K}].\vec{P}}
%objeto general
\newcommand{\genobj} {(I,J) \rhd [l_{1}:(\widetilde{x_{1}�})\vec{P_1}, \ldots , l_{m}:(\widetilde{x_{n}�})\vec{P_m}]}

%mensaje a J general
\newcommand{\genmessj}{�J \lhd l_{i}�:[\widetilde{K}].\vec{P}}
%mensaje a J prime general
\newcommand{\genmessjprime}{�J' \lhd l_{i}�:[\widetilde{K}].\vec{P}}
%mensaje nuevo Pico
\newcommand{\genSF}{(\phi sender,\delta forward) \triangleright [l_1:(\widetilde{x_1})\vec{P_1},...,l_{m}:(\widetilde{x_{m}})\vec{P_{m}}]}
%H despu'es de crear J
\newcommand{\hSF}{(\widetilde{x_{i}} \rightarrow \widetilde{L})+(J \rightarrow L)} 
% mensaje delegacion
\newcommand{\menDel}{(J\triangleleft l_{i}: [\widetilde {K}] \ then \ Q)}



%\newtheorem{teo}{Teorema}[section] 
%\newtheorem{df}[teo]{Definici'on}
%\newtheorem{ex}[teo]{Ejemplo} 
%\newtheorem{prop}[teo]{Proposici'on} 
%\newtheorem{lemma}[teo]{Lema} 
 
%[|  |]
\newcommand{\ozexec}[1]{[\hspace{-0.06cm}[{#1}]\hspace{-0.06cm}]}
\newcommand{\ozel}{[\hspace{-0.06cm}[}
\newcommand{\ozer}{]\hspace{-0.06cm}]}

% ******************** EOF ***************************************







