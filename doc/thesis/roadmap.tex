
\chapter{Roadmap}
\label{roadmap}

Un RoadMap (que podría traducirse como hoja de ruta) es una planificación del desarrollo de un software con los objetivos a corto y largo plazo. A continuación se da una visión general de hacia adónde apunta VirtShell en el futuro:

\begin{itemize}
\item Implementar una interfaz web que permita administrar los ambientes y maquinas virtuales.
\item Implementar algun mecanismo de seguridad que permita revisar las tramas que llegan y salen de las maquinas virtuales y los hosts.
\item Realizar un plan de pruebas funcionales para los ambientes que se aprovisionan.
\item Validar los datos de entrada de los json, tipos y campos mandatorios.
\item Cambiar la forma de seleccionar un host para que tenga en cuenta las métricas e información del sistema de los anfitriones candidatos.
\item Implementar scripts que permitan el despliegue del servidor de VirtShell en uno o mas servidores con balanceadores de carga.
\item Integrar VirtShell con diferentes nubes privadas como Amazon.
\item Mejorar la separación de la base de datos en varios servidores, implementando una capa de abstracción que permita el ruteo dinámico de los datos.
\item Crear el servicio que proporcione monitorización para las instancias.
\item Crear el servicio de auto scaling que permita escalar automaticamente las instancias en función de politicas definidas.
\item Agregar capacidad al módulo de tareas para que diferentes acciones del framework puedan ser programadas como trabajos que se deban ejecutar por la ocurrencia de un evento, o en un tiempo especifico. 
\item Extender el servicio de creación de imágenes desatendidas para soportar diferentes distribuciones de sistemas operativos.
\end{itemize}
        