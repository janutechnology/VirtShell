\subsection{Partitions}
Las particones permiten organizar las máquinas que albergaran recursos virtuales en partes aisladas de las demás.Los métodos soportados son:

\begin{center}
 \captionof{table}{Métodos HTTP para partitions}
 \begin{tabular}{| l | l | l | l |}
 \hline
  \rowcolor{blueapi}
  \textbf{Acci'on} & \textbf{Método HTTP} & \textbf{Solicitud HTTP} & \textbf{Descripci'on} \\ [0.5ex] 
  \hline\hline
  get & GET & /partitions/:name & Gets one partition by name. \\
  \hline
  list & GET & /partitions & Retrieves the list of partitions. \\
  \hline  
  create & POST & /partitions/ & Inserts a new partition configuration. \\
  \hline
  delete & DELETE & /partitions/:name & Deletes an existing partition. \\
  \hline  
  update & PUT & /partitions/:name/host/:hostname & Add a host to partition. \\ [1ex] 
  \hline
\end{tabular}
\end{center}

Representaci'on del recurso de una partición:

\medskip
\begin{lstlisting}[style=json]
{
  "uuid": string,
  "name": string,
  "description": string, 
  "hosts": [ ... list of hosts associated with the Partitions ...],
  "created": {"at": number, "by": string},
  "modified": {"at": number, "by": string}
}
\end{lstlisting}

Ejemplo:

\medskip
\begin{lstlisting}[style=json]
{
  "uuid": "ab8076c0-db91-11e2-82ce-0002a5d5c51b",
  "name": "development_co",
  "description": "Collection of servers oriented to development team in Colombia.", 
  "hosts": [ ... list of hosts associated with the partition ...],
  "created": {"at":"1429207233", "by":"92d30f0c-8c9c-11e5-8994-feff819cdc9f"},
  "modified": {"at":"1529207233", "by":"92d31132-8c9c-11e5-8994-feff819cdc9f"}
}
\end{lstlisting}

\subsubsection{Ejemplos de peticiones HTTP}

\paragraph{Crear una nueva partición - POST /api/virtshell/v1/partitions} ~\\

\begin{lstlisting}[style=json]
curl -sv -X POST \
  -H 'accept: application/json' \
  -H 'X-VirtShell-Authorization: UserId:Signature' \
  -d '{
       "name": "development_co",
       "description": "Collection of servers oriented to development team in colombia."
      }' \
   'http://localhost:8080/api/virtshell/v1/partitions'
\end{lstlisting}

Response:

\begin{lstlisting}[style=json]
HTTP/1.1 200 OK
Content-Type: application/json
{ "create": "success" }
\end{lstlisting}

\paragraph{Obtener una partición- GET /api/virtshell/v1/partitions/:name} ~\\

\begin{lstlisting}[style=json]
curl -sv -H 'accept: application/json' 
     -H 'X-VirtShell-Authorization: UserId:Signature' \ 
     'http://<host>:<port>/api/virtshell/v1/partitions/development_co'
\end{lstlisting}

Response:

\begin{lstlisting}[style=json]
HTTP/1.1 200 OK
Content-Type: application/json
{
  "uuid": "efa1777c-cad7-11e5-9956-625662870761",
  "name": "backend_development_04",
  "description": "Servers for backend of the company", 
  "hosts": [ ... list of hosts associated with the section ...],  
  "created": {"at":"1429207233", "by":"1a900cdc-cad8-11e5-9956-625662870761"},
  "modified": {"at":"1529207233", "by":"2163b554-cad8-11e5-9956-625662870761"}
}
\end{lstlisting}

\paragraph{Obtener todas las particiones - GET /api/virtshell/v1/partitions} ~\\

\begin{lstlisting}[style=json]
curl -sv -H 'accept: application/json' 
     -H 'X-VirtShell-Authorization: UserId:Signature' \ 
     'http://localhost:8080/api/virtshell/v1/partitions'
\end{lstlisting}

Response:

\begin{lstlisting}[style=json]
HTTP/1.1 200 OK
Content-Type: application/json
{
  "partitions": [
    {
      "uuid": "ab8076c0-db91-11e2-82ce-0002a5d5c51b",
      "name": "development_co",
      "description": "Collection of servers oriented to development team in colombia.",
      "hosts": [ ... list of hosts associated with the section ...],
      "created": {"at":"1429207233", "by":"92d30f0c-8c9c-11e5-8994-feff819cdc9f"},
      "modified": {"at":"1529207233", "by":"92d31132-8c9c-11e5-8994-feff819cdc9f"}
    },
    { 
      "uuid": "efa1777c-cad7-11e5-9956-625662870761",
      "name": "production_us_miami",
      "description": "Collection of servers oriented to production in us.",
      "hosts": [ ... list of hosts associated with the section ...],      
      "created": {"at":"1429207233", "by":"1a900cdc-cad8-11e5-9956-625662870761"},
      "modified": {"at":"1529207233", "by":"2163b554-cad8-11e5-9956-625662870761"}
    }    
  ]
}  
\end{lstlisting}

\paragraph{Eliminar una partición - DELETE /api/virtshell/v1/partitions/:name} ~\\

\begin{lstlisting}[style=json]
curl -sv -X DELETE \
   -H 'accept: application/json' \
   -H 'X-VirtShell-Authorization: UserId:Signature' \
   'http://<host>:<port>/api/virtshell/v1/partitions/backend_development_04'
\end{lstlisting}

Response:

\begin{lstlisting}[style=json]
HTTP/1.1 200 OK
Content-Type: application/json
```
```json
{ "delete": "success" }
\end{lstlisting}

\paragraph{Agregar un host a una partición - PUT /api/virtshell/v1/partitions/:name/host/:hostname} ~\\

\begin{lstlisting}[style=json]
curl -sv -X PUT \
  -H 'accept: application/json' \
  -H 'X-VirtShell-Authorization: UserId:Signature' \
  'http://localhost:8080/virtshell/api/v1/partitions/:name/host/:hostname'
\end{lstlisting}

Response:

\begin{lstlisting}[style=json]
HTTP/1.1 200 OK
Content-Type: application/json

{ "add_host": "success" }
\end{lstlisting}