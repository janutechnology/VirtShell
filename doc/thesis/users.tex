\subsection{Users}
Representan los usuarios registrados en VirtShell. Los metodos soportados son:

\begin{center}
 \begin{tabular}{| l | l | l | l |}
 \hline
  \rowcolor{blueapi}
  \textbf{Acci'on} & \textbf{Metodo HTTP} & \textbf{Solicitud HTTP} & \textbf{Descripci'on} \\ [0.5ex] 
  \hline\hline
  get & GET & /users/id & Gets one user by ID. \\
  \hline
  create & POST & /users/ & creates a new user. \\
  \hline
  list & GET & /users & Retrieves the list of users. \\  
  \hline
  delete & DELETE & /users/id & Deletes an existing user. \\
  \hline  
  update & PUT & /users/id & Updates an existing user. \\ [1ex]  
  \hline
\end{tabular}
\end{center}

\vspace{1cm}
Representaci'on del recurso de un usuario:
\vspace{1cm}

\begin{lstlisting}[style=json]
{
  "uuid": "ab8076c0-db91-11e2-82ce-0002a5d5c51b",
  "username": "virtshell",
  "type": "system/regular",
  "login": "user@mail.com",
  "groups": [ ... list of users ...],
  "created": {"at": timestamp, "by": user_uuid},
  "modified": {"at": timestamp, "by": user_uuid}
}
\end{lstlisting}

Ejemplo:

\medskip
\begin{lstlisting}[style=json]
{
  "uuid": "ab8076c0-db91-11e2-82ce-0002a5d5c51b",
  "username": "virtshell",
  "type": "system/regular",
  "login": "user@mail.com",
  "groups": [ {"uuid": "a146cae4-8c90-11e5-8994-feff819cdc9f"},
              {"uuid": "a146d00c-8c90-11e5-8994-feff819cdc9f"}
  ],
  "created": {"at":"1429207233", "by":"92d30f0c-8c9c-11e5-8994-feff819cdc9f"},
  "modified": {"at":"1529207233", "by":"92d31132-8c9c-11e5-8994-feff819cdc9f"}
}
\end{lstlisting}

\subsubsection{Ejemplos de peticiones HTTP}

\paragraph{Crear un nuevo usuario - POST /api/virtshell/v1/users} ~\\

\begin{lstlisting}[style=json]
curl -X POST \
  -H 'accept: application/json' \
  -H 'X-VirtShell-Authorization: UserId:Signature' \
  -H "Content-Type: multipart/form-data" \
  -d {"uuid": "ab8076c0-db91-11e2-82ce-0002a5d5c51b",
       "username": "virtshell", 
       "type": "system/regular",
       "login": "user@mail.com",
       "groups": [ {"uuid": "a146cae4-8c90-11e5-8994-feff819cdc9f"},
                   {"uuid": "a146d00c-8c90-11e5-8994-feff819cdc9f"}
        ],
       "created": {"at":"1429207233", "by":"92d30f0c-8c9c-11e5-8994-feff819cdc9f"},
       "modified": {"at":"1529207233", "by":"92d31132-8c9c-11e5-8994-feff819cdc9f"}
      } \
  'http://<host>:<port>/api/virtshell/v1/users'
\end{lstlisting}

\vspace{1cm}
Respuesta:
\vspace{1cm}

\begin{lstlisting}[style=json]
HTTP/1.1 200 OK
Content-Type: application/json
{ "create": "success" }
\end{lstlisting}

\paragraph{Obtener un usuario - GET /api/virtshell/v1/users/:id} ~\\

\begin{lstlisting}[style=json]
curl -sv -H 'accept: application/json' 
     -H 'X-VirtShell-Authorization: UserId:Signature' \ 
     'http://<host>:<port>/api/virtshell/v1/users/?id=ab8076c0-db91-11e2-82ce-0002a5d5c51b'
\end{lstlisting}

\vspace{1cm}
Respuesta:
\vspace{1cm}

\begin{lstlisting}[style=json]
HTTP/1.1 200 OK
Content-Type: application/json
{
  "uuid": "ab8076c0-db91-11e2-82ce-0002a5d5c51b",
  "username": "virtshell",
  "type": "system/regular",
  "login": "user@mail.com",
  "groups": [ {"uuid": "a146cae4-8c90-11e5-8994-feff819cdc9f"}],
  "created": {"at":"1429207233", "by":"92d30f0c-8c9c-11e5-8994-feff819cdc9f"},
  "modified": {"at":"1529207233", "by":"92d31132-8c9c-11e5-8994-feff819cdc9f"}
}
\end{lstlisting}

\paragraph{Actualizar un usuario - PUT /api/virtshell/v1/users/:id} ~\\

\begin{lstlisting}[style=json]
curl -sv -X PUT \
  -H 'accept: application/json' \
  -H 'X-VirtShell-Authorization: UserId:Signature' \
  -H "Content-Type: multipart/form-data" \
  -d '{"type": "system",
       "groups": [{"uuid": "a146cae4-8c90-11e5-8994-feff819cdc9f"},
                  {"uuid": "a146d00c-8c90-11e5-8994-feff819cdc9f"}]}' \
   'http://localhost:8080/api/virtshell/v1/file?id=8de7b824-d7d1-4265-a3a6-5b46cc9b8ed5'
\end{lstlisting}

\vspace{1cm}
Respuesta:
\vspace{1cm}

\begin{lstlisting}[style=json]
HTTP/1.1 200 OK
Content-Type: application/json

{ "update": "success" }
\end{lstlisting}


\paragraph{Eliminar un usuario - DELETE /api/virtshell/v1/users/:id} ~\\

\begin{lstlisting}[style=json]
curl -sv -X DELETE \
   -H 'accept: application/json' \
   -H 'X-VirtShell-Authorization: UserId:Signature' \
   'http://localhost:8080/api/virtshell/v1/fles?id=ab8076c0-db91-11e2-82ce-0002a5d5c51b'
\end{lstlisting}

\vspace{1cm}
Respuesta:
\vspace{1cm}

\begin{lstlisting}[style=json]
HTTP/1.1 200 OK
Content-Type: application/json
```
```json
{ "delete": "success" }
\end{lstlisting}


% VirtShell is a multi-user framework that is based on the Unix permissions concepts to provide security.

% VirtShell provides mechanisms to control access by  limiting the types of
% resource access that can be made. Access is permitted or denied depending on
% several factors, one of which is the type of access requested. Several different
% types of operations may be controlled:

% Read. Read from the resouce.
% Write. Write or rewrite of resoures.
% Execute. Load the resource into host and execute it.

% Here is a quick breakdown of the access that the three basic permission types grant a user.

% Read
% ----
% Read permission allows a user to view the contents of any resource in VirtShell.

% Write
% -----
% Write permission allows a user to create, modify and delete whatever resources.

% Execute
% -------
% Execute permission allows a user to execute virtual machines or containers, for example: start, stop, pause, snapshot. (the user must also have read permission). 