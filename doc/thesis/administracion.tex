\chapter{Adminsitración}
\label{capadministracion}

La capa de Administración de VirtShell proporciona una infraestructura de servicios para la gestión de cualquier dispositivo registrado en el sistema. Este capitulo busca darle explicación a las funcionalidades de administración para utilizarlos en su beneficio.

\section{Particiones y Ambientes en VirtShell}
En VirtShell hay dos conceptos que son muy importantes y se extiende a través de todos los servicios, y que simplemente no puede dejar de tener en cuenta: Las particiones y los ambientes. Ambos se asocian con la mayoría de las cosas en VirtShell, y el dominio de ellos es crucial para una buena administración de los dispositivos. 

\subsection{Particiones}
- Particiones pueden estar dispersas geograficamente
- 


\subsection{Division de partciones en ambientes}
Un ambiente es un lugar aislado dentro de una partición. Cada partición está compuesta por varios ambientes. Cada ambiente pertenece a una sola región. La siguiente imagen explica el concepto:

\section{Instancias}
- el api ofrece funcionalidad para empezar y parar instancias de servidor, aplicar permisos de acceso y red o gestionar tus imágenes de servidor.


