\chapter{Adminsitración}
\label{capadministracion}

La capa de Administración de VirtShell proporciona una infraestructura de servicios para la gestión de cualquier dispositivo registrado en el sistema. Este capítulo busca darle explicación a las funcionalidades de administración para utilizarlas en su beneficio.

\section{Particiones, anfitriones y Ambientes en VirtShell}
En VirtShell hay tres conceptos que son muy importantes y se extiende a través de todos los servicios, y que simplemente no puede dejar de tener en cuenta: Las particiones, los ambientes y las instancias. Las tres se asocian con la mayoría de las cosas en VirtShell, y el dominio de ellos es crucial para una buena administración de los dispositivos. 

\subsection{Particiones}
Las particiones consisten de uno o más anfitriones, los cuales pueden ser nodos físicos, servidores o incluso máquinas virtuales. El objetivo principal que busca una partición, es organizar las máquinas que albergaran recursos virtuales en partes aisladas de las demás. Estas partes pueden pueden estar ubicadas en un mismo sitio físico o por el contrario puede estar distribuidas es diferentes zonas geográficas de todo el mundo.\\
\\
Si solo se cuenta con un numero fijo de maquinas (o anfitriones) ubicadas en un mismo sitio físico como por ejemplo un datacenter \footnote{Un data center también llamado centro de datos es un espacio acondicionado especialmente para contener a todos los equipos y sistemas de TI}, lo que se obtiene con las particiones es la posibilidad de dividir esas maquinas en subgrupos que puedan ser destinados para diferentes equipos o divisiones dentro de una organización.\\ 
\\
Al contar con maquinas distribuidas en diferentes zonas geográficas la elección de una partición u otra se basa principalmente en la cercanía de los visitantes o clientes, ya que a menor distancia entre los servidores y ellos, menores son los tiempos de respuesta y mejor la experiencia de usuario.\\
\\
Las particiones también favorecen la disponibilidad. Si distribuye sus instancias a través de múltiples particiones y una instancia falla, puede diseñar su aplicación para que una instancia en otra partición pueda atender las peticiones.\\
\\
Cuando se crea una nueva partición, VirtShell la crea completamente vaciá, sin anfitriones. Para asociar anfitriones a una partición se debe crear un anfitrión y vincularlo con la partición como se vera mas adelante en este mismo capítulo. Un ejemplo de como crear una partición usando el API se muestra en el siguiente código:

\begin{lstlisting}[style=json, caption=Petición HTTP para crear una partición]
curl -sv -X POST \
  -H 'accept: application/json' \
  -H 'X-VirtShell-Authorization: UserId:Signature' \
  -d '{
  	   "name": "development_co",
       "description": "Collection of servers oriented to development team in colombia."
      }' \
   'http://localhost:8080/api/virtshell/v1/partitions'
\end{lstlisting}


\subsection{Asociación de anfitriones a particiones}
Los anfitriones no son mas que nodos físicos, servidores o máquinas virtuales, que alojaran recursos virtuales. VirtShell ofrece la posibilidad de clasificarlos de acuerdo a combinaciones de capacidad de CPU, memoria, almacenamiento y red. El objetivo que busca la clasificación es proporcionar flexibilidad para elegir la combinación de recursos adecuada para las aplicaciones.\\
\\
Los tipos de anfitriones se agrupan en familias basadas en perfiles de aplicación de destino. Estos grupos incluyen: de propósito general, con procesadores de alto desempeño, de memoria optimizada, de almacenamiento optimizado.

\begin{description}
\item [Propósito general] Esta familia proporciona un equilibrio de recursos informáticos, de memoria y red, por lo que constituye una buena opción para muchas aplicaciones.
\item [Procesadores de alto desempeño] Esta familia ofrece procesadores que alcanzan alto desempeño en tareas complejas.
\item [Memoria optimizada] Esta familia esta optimizada para aplicaciones con un uso intenso de la memoria.
\item [Almacenamiento optimizado] Esta familia promete anfitriones con alta capacidad de almacenamiento, optimizado para un desempeño de E/S muy alto.
\end{description}

Cuando se crea un anfitrión en VirtShell este debe ser asociado a una sola partición, asignándole un tipo, con alguno de los mencionados anteriormente, estableciendo las capacidad de disco y memoria RAM con las que cuenta,  
indicando también el sistema operativo y las ip con las que se conecta a la red. Una vez el anfitrión es creado en el sistema este queda asignado a la partición elegida. El siguiente ejemplo muestra como crear un anfitrión usando el API:

\begin{lstlisting}[style=json, caption=Petición HTTP para crear un host]
curl -sv -X POST \
  -H 'accept: application/json' \
    -H 'X-VirtShell-Authorization: UserId:Signature' \
  -d '{"name": "host-01-pdn",
       "os": "Ubuntu_12.04_3.5.0-23.x86_64",
       "memory": "16GB",
       "capacity": "120GB",
       "enabled": "true",
       "type" : "GeneralPurpose",
       "local_ipv4": "15.54.88.19",
       "local_ipv6": "ff06:0:0:0:0:0:0:c3",
       "public_ipv4": "10.54.88.19",
       "public_ipv6": "yt06:0:0:0:0:0:0:c3",
       "partition": "development_co"}' \
   'http://localhost:8080/virtshell/api/v1/hosts'
\end{lstlisting}

En este ejemplo se muestra la creación de un host con nombre host-01-pdn que esta clasificado como de uso general y que queda asociado a la partición development\_co creada en la sección anterior.\\
\\
Al consultar la partición nuevamente por medio del API se puede observar como el anfitrión se encuentra asociado a la partición developtment\_co. En la siguiente consulta al API se muestra el resultado de la asociación: 

\begin{lstlisting}[style=json, , caption=Petición HTTP para consultar una partición por su nombre]
curl -sv -H 'accept: application/json' 
     -H 'X-VirtShell-Authorization: UserId:Signature' \ 
     'http://<host>:<port>/api/virtshell/v1/partitions/development_co'
\end{lstlisting}

Response:

\begin{lstlisting}[style=json]
HTTP/1.1 200 OK
Content-Type: application/json
{
  "uuid": "efa1777c-cad7-11e5-9956-625662870761",
  "name": "development_co",
  "description": "Collection of servers oriented to development team in colombia.", 
  "hosts": [ "host-01-pdn" ],  
  "created": {"at":"1429207233", 
              "by":"1a900cdc-cad8-11e5-9956-625662870761"}
}
\end{lstlisting}

Adicionalmente, cuando un anfitrión es agregado a una partición, VirtShell instala automáticamente los agentes  que realizaran tareas de aprovisionamiento y monitoreo. Los agentes se explicaran mas adelante.

\subsection{División de particiones en ambientes}
Las particiones se refieren a la forma de organizar lugares físicos. Los ambientes por el contrario, son lugares lógicos y aislados dentro de una partición. Cada partición puede estar compuesta por uno o varios ambientes, y un ambiente puede contener uno o mas instancias. Cada ambiente pertenece a una sola partición. La figura \ref{fig:enviroment} muestra un ejemplo de como dos particiones contiene varios ambientes destinados a equipos de trabajo diferentes. \\

\begin{figure}[h]
    \centering
	\caption{Ejemplo de ambientes en un partición}
	\label{fig:enviroment}
	\includegraphics[width = 0.95\textwidth]{../architecture/v1/diagrams/enviroments}
\end{figure}

Un ejemplo de como crear un ambiente asociado a una partición usando el API se muestra en el siguiente código:

\begin{lstlisting}[style=json, caption=Petición HTTP para crear un ambiente]
curl -sv -X POST \
  -H 'accept: application/json' \
  -H 'X-VirtShell-Authorization: UserId:Signature' \
  -d '{
       "name": "bigdata_laboratory",
       "description": "Collection of servers oriented to big data.", 
       "partition": "bogota_partition_co"
      }' \
   'http://localhost:8080/api/virtshell/v1/enviroments'
\end{lstlisting}

\section{Instancias}
A un servidor virtual en VirtShell, se le denomina instancia. Las instancias pueden ser maquinas virtuales que corren sobre algún hipervisor o también pueden ser contenedores que se ejecutan directamente sobre el sistema operativo del anfitrión. La elección de la tecnología de virtualización depende de las preferencias u objetivos que se busque con las aplicaciones de la instancia.\\
\\
Las instancias son designadas a un particular ambiente de trabajo en el momento de ser creadas. En el momento de su creación se debe especificar las características de CPU, memoria y capacidad de disco que se desea. De igual manera, se configura el sistema operativo, los permisos para los demás usuarios del sistema, el tipo de anfitrión que necesita y la tecnología que se desea usar para virtualizar.\\ 
\\
\subsection{Creación de instancias en un ambiente}
Cuando se recibe una solicitud de creación de una instancia, VirtShell selecciona uno anfitrión de los que se encuentran configurados en la partición a la cual pertenece el ambiente definido en la petición, y de acuerdo al tipo de anfitrión solicitado. Si la partición no cuenta con el tipo de anfitrión solicitado, se rechazara la petición detallando el inconveniente. Si por el contrario la selección del anfitrión es exitosa, VirtShell procede a enviar la solicitud al agente de aprovisionamiento del anfitrión escogido. \\
\\
Debido a que una solicitud de aprovisionamiento puede involucrar mas de una instancia, el servidor de VirtShell responde la solicitud de aprovisionamiento con el estado de recibido y con el identificador de una tarea. La tarea será ejecutada de manera asincrónica y el usuario por medio del API REST puede consultar el estado y avance de la misma.\\
\\
Un ejemplo de la creación de una instancia usando el API se muestra en el siguiente código:

\begin{lstlisting}[style=json, caption=Petición HTTP para crear un ambiente]
curl -sv -X POST \
  -H 'accept: application/json' \
  -H 'X-VirtShell-Authorization: UserId:Signature' \
  -d '{ "name": "transactional_log",
        "memory": 1024,
        "cpus": 2,
        "hdsize": "2GB",
        "operating_system": "ubuntu_server_14.04.2_amd64",
        "description": "Server transactional only for store logs", 
        "provisioner": "all_backend",
        "host_type": "GeneralPurpose",
        "driver": "lxc"
      }' \
   'http://localhost:8080/api/virtshell/v1/instances'
\end{lstlisting}



