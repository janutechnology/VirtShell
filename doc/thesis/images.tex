\section{Images}
Representan imagenes de m'aquinas virtuales o contenedores. Los metodos soportados son:

\begin{center}
 \begin{tabular}{| l | l | l | l |}
 \hline
  \rowcolor{blueapi}
  \textbf{Acci'on} & \textbf{Metodo HTTP} & \textbf{Solicitud HTTP} & \textbf{Descripci'on} \\ [0.5ex] 
  \hline\hline
  get & GET & /images/id & Gets one image by ID. \\
  \hline
  list & GET & /images & Retrieves the list of images. \\
  \hline  
  create & POST & /images/ & Inserts a new image. \\
  \hline
  delete & DELETE & /images/id & Deletes an existing image. \\
  \hline  
  update & PUT & /images/id & Updates an existing image. \\ [1ex] 
  \hline
\end{tabular}
\end{center}

\vspace{1cm}
Representaci'on del recurso de una imagen:
\vspace{1cm}

\begin{lstlisting}[style=json]
{
  "id": "kj5436c0-dc94-13tg-82ce-9992b5d5c51b",
  "name": "ubuntu_server_14.04.2_amd64",
  "type": "iso|container",
  "os": "ubuntu", 
  "release": "trusty",
  "version": "14.04.2", 
  "variant": "server|desktop", 
  "arch": "i386|amd64", 
  "timezone": "America/Bogota", 
  "ssh_key": "-------- BEGIN PUBLIC KEY ----and a valid key here",
  "preseed_file": "seed_file_uuid",
  "url_image": "https://gist.github.com/hagix9/3514296#file-lxc-centos",
  "path_image": "path_image_uuid",  
  "created":["at":"20130625105211","by":10],
  "details": "...."
}
\end{lstlisting}

Ejemplo:

\medskip
\begin{lstlisting}[style=json]
{
  "id": "k05436c0-cc94-13tg-82cb-9992b5d5c51b",
  "name": "ubuntu_server_14.04.2_amd64",
  "type": "iso",
  "os": "ubuntu", 
  "release": "trusty",
  "version": "14.04.2", 
  "variant": "server", 
  "arch": "amd64", 
  "timezone": "America/Bogota", 
  "key": "ssh-rsa AAAAB3NzaC1yc2EBBBBDAQABAAABAQC2Mo6tWWf6FYIBNz8tHgbfTiYkOC3++ToKe8g5FjNW9Uw86CeqmP4PiTZiQwdDCDqikk9xETEEhVnjDEJO4mx8W/q77Ciq1wGnraqI9FfNfH6LRfGnZ+rHxr371pbcqHXFY11WatdyjvmPiyDmMWTJzoDUXANIQ5YE9Lpb23PwtUQ3FzyBtImGA+i474Vf/Opz68kSzEElm+oaNmLNcHp0AyomA6i5xTTxYyP2+BEJtVT0CSXM5YVbs8iXVlD/3XdukT9J0oFzdxKkI5lBvhnhjE91XnfllNuE9gAoZGiX3Cya384ofNGwe8ARu9Wi29a4zoRcMpV8AD0TLYRHeJJJ yourcompany@gmail.com",
  "preseed_file": "fe1bb90a-7f25-11e5-8bcf-feff819cdc9f",
  "created":["at":"20130625105211","by":10]
}
\end{lstlisting}

\subsection{Ejemplos de peticiones HTTP}

\subsubsection{Crear un nuevo host - POST /virtshell/api/v1/images}

\begin{lstlisting}[style=json]
curl -sv -X POST \
  -H 'accept: application/json' \
    -H 'X-VirtShell-Authorization: UserId:Signature' \
  -d '{"name": "host-01-pdn",
       "os": "Ubuntu_12.04_3.5.0-23.x86_64",
       "memory": "16GB",
       "capacity": "120GB",
       "enabled": "true",
       "type" : "GeneralPurpose",
       "local_ipv4": "15.54.88.19",
         "local_ipv6": "ff06:0:0:0:0:0:0:c3",
       "public_ipv4": "10.54.88.19",
       "public_ipv6": "yt06:0:0:0:0:0:0:c3"}' \
   'http://localhost:8080/virtshell/api/v1/hosts'
\end{lstlisting}

\vspace{1cm}
Respuesta:
\vspace{1cm}

\begin{lstlisting}[style=json]
HTTP/1.1 200 OK
Content-Type: application/json
{ "create": "success" }
\end{lstlisting}

\subsubsection{Obtener un host- GET /virtshell/api/v1/images/:id}

\begin{lstlisting}[style=json]
curl -sv -H 'accept: application/json' 
     -H 'X-VirtShell-Authorization: UserId:Signature' \ 
     'http://localhost:8080/api/virtshell/v1/hosts?id=ab8076c0-db91-11e2-82ce-0002a5d5c51b'
\end{lstlisting}

\vspace{1cm}
Respuesta:
\vspace{1cm}

\begin{lstlisting}[style=json]
HTTP/1.1 200 OK
Content-Type: application/json
{
  "uuid": "ab8076c0-db91-11e2-82ce-0002a5d5c51b",
  "name": "host-01-pdn",
  "os": "Ubuntu_12.04_3.5.0-23.x86_64",
  "memory": "16GB",
  "capacity": "120GB",
  "enabled": "true",
  "type" : "StorageOptimized",
  "local_ipv4": "15.54.88.19",
  "local_ipv6": "ff06:0:0:0:0:0:0:c3",
  "public_ipv4": "10.54.88.19",
  "public_ipv6": "yt06:0:0:0:0:0:0:c3",
  "instances": [
    {
      "name": "name1",
      "id": "72C05559-0590-4DA6-BE56-28AB36CB669C"
    },
    {
      "name": "name2",
      "id": "17173587-C4E9-4369-9C43-FCBF5E075973"
    }
  ],
  "created":["at":"20130625105211", "by":10]
}
\end{lstlisting}

\subsubsection{Obtener todos los host - GET /virtshell/api/v1/images}

\begin{lstlisting}[style=json]
curl -sv -H 'accept: application/json' 
     -H 'X-VirtShell-Authorization: UserId:Signature' \ 
     'http://localhost:8080/api/virtshell/v1/hosts'
\end{lstlisting}

\vspace{1cm}
Respuesta:
\vspace{1cm}

\begin{lstlisting}[style=json]
HTTP/1.1 200 OK
Content-Type: application/json
{
  "hosts": [
    {
      "uuid": "ab8076c0-db91-11e2-82ce-0002a5d5c51b",
      "name": "host-01-pdn",
      "os": "Ubuntu_12.04_3.5.0-23.x86_64",
      "memory": "16GB",
      "capacity": "120GB",
      "enabled": "true",
      "type" : "StorageOptimized",
      "local_ipv4": "15.54.88.19",
      "local_ipv6": "ff06:0:0:0:0:0:0:c3",
      "public_ipv4": "10.54.88.19",
      "public_ipv6": "yt06:0:0:0:0:0:0:c3",
      "instances": [
        {
          "name": "name1",
          "id": "72C05559-0590-4DA6-BE56-28AB36CB669C"
        },
        {
          "name": "name2",
          "id": "17173587-C4E9-4369-9C43-FCBF5E075973"
        }
      ],
      "created":["at":"20130625105211", "by":10]
    },
    {
      "uuid": "ab8076c0-db91-11e2-82ce-0002a5d5c51b",
      "name": "host-01-pdn",
      "os": "Ubuntu_12.04_3.5.0-23.x86_64",
      "memory": "16GB",
      "capacity": "120GB",
      "enabled": "true",
      "type" : "GeneralPurpose",
      "local_ipv4": "15.54.88.19",
      "local_ipv6": "ff06:0:0:0:0:0:0:c3",
      "public_ipv4": "10.54.88.19",
      "public_ipv6": "yt06:0:0:0:0:0:0:c3",
      "instances": [
        {
          "name": "name3",
          "id": "DE11CC9A-482F-4033-A7F8-503EE449DD0A"
        },
        {
          "name": "name4",
          "id": "17173587-C4E9-4369-9C43-FCBF5E075973"
        },    
      ],
      "created":["at":"20130625105211", "by":10]
    }
  ]
}   
\end{lstlisting}

\subsubsection{Actualizar un host - PUT /virtshell/api/v1/images/:id}

\begin{lstlisting}[style=json]
curl -sv -X PUT \
  -H 'accept: application/json' \
    -H 'X-VirtShell-Authorization: UserId:Signature' \
  -d '{"memory": "24GB",
     "capacity": "750GB"}' \
   'http://localhost:8080/api/virtshell/v1/hosts?id=ab8076c0-db91-11e2-82ce-0002a5d5c51b'
\end{lstlisting}

\vspace{1cm}
Respuesta:
\vspace{1cm}

\begin{lstlisting}[style=json]
HTTP/1.1 200 OK
Content-Type: application/json

{ "update": "success" }
\end{lstlisting}

\subsubsection{Eliminar un host - DELETE /virtshell/api/v1/images/:id}

\begin{lstlisting}[style=json]
curl -sv -X DELETE \
   -H 'accept: application/json' \
   -H 'X-VirtShell-Authorization: UserId:Signature' \
   'http://localhost:8080/api/virtshell/v1/hosts?id=ab8076c0-db91-11e2-82ce-0002a5d5c51b'
\end{lstlisting}

\vspace{1cm}
Respuesta:
\vspace{1cm}

\begin{lstlisting}[style=json]
HTTP/1.1 200 OK
Content-Type: application/json
```
```json
{ "delete": "success" }
\end{lstlisting}
